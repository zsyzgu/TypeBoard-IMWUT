%%
%% This is file `sample-acmlarge.tex',
%% generated with the docstrip utility.
%%
%% The original source files were:
%%
%% samples.dtx  (with options: `acmlarge')
%% 
%% IMPORTANT NOTICE:
%% 
%% For the copyright see the source file.
%% 
%% Any modified versions of this file must be renamed
%% with new filenames distinct from sample-acmlarge.tex.
%% 
%% For distribution of the original source see the terms
%% for copying and modification in the file samples.dtx.
%% 
%% This generated file may be distributed as long as the
%% original source files, as listed above, are part of the
%% same distribution. (The sources need not necessarily be
%% in the same archive or directory.)
%%
%% The first command in your LaTeX source must be the \documentclass command.
\documentclass[acmlarge]{acmart}

\def\red#1{\textcolor[rgb]{1,0,0}{#1}}
\usepackage{float}
\usepackage{CJKutf8}
\usepackage{subcaption}
\usepackage{multirow}
\usepackage{url}
\usepackage{appendix}
\usepackage{array, multirow, multicol, rotating, makecell, caption, booktabs}
\usepackage{xcolor,colortbl}

%%
%% \BibTeX command to typeset BibTeX logo in the docs
\AtBeginDocument{%
  \providecommand\BibTeX{{%
    \normalfont B\kern-0.5em{\scshape i\kern-0.25em b}\kern-0.8em\TeX}}}

%% Rights management information.  This information is sent to you
%% when you complete the rights form.  These commands have SAMPLE
%% values in them; it is your responsibility as an author to replace
%% the commands and values with those provided to you when you
%% complete the rights form.
\setcopyright{acmcopyright}
\copyrightyear{2018}
\acmYear{2018}
\acmDOI{10.1145/1122445.1122456}


%%
%% These commands are for a JOURNAL article.
\acmJournal{POMACS}
\acmVolume{37}
\acmNumber{4}
\acmArticle{111}
\acmMonth{8}

%%
%% Submission ID.
%% Use this when submitting an article to a sponsored event. You'll
%% receive a unique submission ID from the organizers
%% of the event, and this ID should be used as the parameter to this command.
%%\acmSubmissionID{123-A56-BU3}

%%
%% The majority of ACM publications use numbered citations and
%% references.  The command \citestyle{authoryear} switches to the
%% "author year" style.
%%
%% If you are preparing content for an event
%% sponsored by ACM SIGGRAPH, you must use the "author year" style of
%% citations and references.
%% Uncommenting
%% the next command will enable that style.
%%\citestyle{acmauthoryear}

%%
%% end of the preamble, start of the body of the document source.
\begin{document}
\begin{CJK*}{UTF8}{gbsn}
		

%%
%% The "title" command has an optional parameter,
%% allowing the author to define a "short title" to be used in page headers.
\title{TypeBoard: Identifying Unintentional Touch on Pressure-Sensitive Touchscreen Keyboards}

%%
%% The "author" command and its associated commands are used to define
%% the authors and their affiliations.
%% Of note is the shared affiliation of the first two authors, and the
%% "authornote" and "authornotemark" commands
%% used to denote shared contribution to the research.
\author{anonymity}
\email{anonymity@anonymity.com}
\affiliation{%
  \institution{address}
  \streetaddress{P.O. Box 1212}
  \city{Dublin}
  \state{Ohio}
  \country{Country}
  \postcode{43017-6221}
}


%%
%% By default, the full list of authors will be used in the page
%% headers. Often, this list is too long, and will overlap
%% other information printed in the page headers. This command allows
%% the author to define a more concise list
%% of authors' names for this purpose.
\renewcommand{\shortauthors}{Trovato and Tobin, et al.}

%%
%% The abstract is a short summary of the work to be presented in the
%% article.
\begin{abstract}

%用户在触屏键盘上打字时,触摸屏幕即触发点击事件。因此,触屏用户不能像在物理键盘上一样通过触摸来align fingers positions[xx],也不能将手指休息在键盘上,影响了触屏打字的效率和舒适度。在这篇论文中,我们提出了TypeBoard,一款带压力屏键盘上的防误触算法,我们也研究了用户使用TypeBoard时的打字行为。用户的打字行为和防误触能力之间是相互影响的,比如,在防误触的触屏键盘上,用户会更倾向于将手指休息在触碰上,造成更多的、更多样的非有意触摸点;而更多的、更多样的非有意触碰点会对防误触提出更高的要求。为此,我们通过迭代的数据采集和机器学习方法来设计了TypeBoard防误触算法。在一个使用TypeBoard写日记的评测实验中,用户的非有意触点个数占触点总数的xx.x\%,我们的算法能在点击事件发生100 ms的时间内以xx.x\%的准确率判断出其输入意图,而相关工作中基于压力和时间阈值的方法[xx]的识别准确率只有xx.x\%,且必须在release以后做出判断。这份工作说明,第一,压敏触屏键盘有能力准确地、低延迟地防止用户休息、轻触等行为造成的误触。第二,用户在有效防误触的键盘上打字时,其用户行为会发生改变,比如手指休息的行为会显著增加。第三,防误触键盘和传统触碰键盘相比,能够防止用户疲劳,提高用户体验,且显著降低了输入任务的完成时间。

Text input is essential in tablet computer interaction. However, tablet software keyboards face the problem of misrecognizing unintentional touch, which affects efficiency and usability \cite{2013-TapBoard, 2018-UbiK}. In this paper, we proposed TypeBoard, a pressure-sensitive touchscreen keyboard that prevents unintentional touches. The TypeBoard allows users to rest their fingers on the touchscreen, which changes the user behavior. On average, users produce 40.83 unintentional touches every 100 keystrokes. The TypeBoard prevents unintentional touch with an accuracy of 98.88\%. A typing study showed that the TypeBoard relieved fatigue ($p< 0.005$), reduced corrected error rate ($p<0.01$), and improved the tablet keyboard' typing speed by 11.78\% ($p<0.005$). 
Under the premise that users could rest their fingers on the touchscreen, we added tactile landmarks on the TypeBoard, allowing users to find the keys through touch rather than sight. The tactile landmarks further improve the typing speed, outperforming the ordinary tablet keyboard by 21.19\% ($p<0.001$).
The results show that pressure-sensitive touchscreen keyboards can prevent unintentional touch, which improves efficiency and usability from many aspects, including relieving fatigue, reducing error rates, and mediately allowing touch typing on the tablet.

%Furthermore, we added stickers on the TypeBoard to provide tactile landmarks, allowing users to find the keys through touch rather than sight. The TypeBoard with tactile landmarks improves the ordinary tablet keyboard' typing speed by 21.19\% ($p<0.001$). In real use, we can add tactile landmarks on the TypeBoard through deformable screens \cite{Website-Tactus}, and changeable surface texture \cite{2011-Stimtac, 2010-TeslaTouch, 2011-Enhancing}.

% 43.7096
% 48.8660
% 52.9708

%触屏软键盘在性能和体验这两方面都无法与物理键盘相提并论。用户可以将手指休息在物理键盘上,但不能休息在软键盘上,否则会引发误触。这种触摸键位而不按下的状态是物理键盘可用性的重要原因。首先,这可以降低疲劳;第二,用户可以通过触摸(而不是视觉)来寻找键位,大大提高了输入效率。为了缩小软键盘和物理键盘之间的差距,我们提出了TypeBoard,一种防误触的软键盘。TypeBoard允许用户将手指休息在触屏上,在此许诺下用户的输入行为会发生变化,每100次有意点击中就会出现xx.x次误触。在这个误触很多的数据集下,我们的模型的预测准去率达到了xx.x\%。实验证明,TypeBoard显著降低了用户打字时的疲劳程度和心理负担。更进一步地,我们在TypeBoard上用贴纸来提供tactile landmarks,从而允许用户通过触觉来寻找键位。实验证明,TypeBoard with landmarks降低了疲劳和心理负担,且将普通软键盘的输入效率显著提高了0.xx倍(F,p)。

\end{abstract}

%%
%% The code below is generated by the tool at http://dl.acm.org/ccs.cfm.
%% Please copy and paste the code instead of the example below.
%%
\begin{CCSXML}
<ccs2012>
 <concept>
  <concept_id>10010520.10010553.10010562</concept_id>
  <concept_desc>Computer systems organization~Embedded systems</concept_desc>
  <concept_significance>500</concept_significance>
 </concept>
 <concept>
  <concept_id>10010520.10010575.10010755</concept_id>
  <concept_desc>Computer systems organization~Redundancy</concept_desc>
  <concept_significance>300</concept_significance>
 </concept>
 <concept>
  <concept_id>10010520.10010553.10010554</concept_id>
  <concept_desc>Computer systems organization~Robotics</concept_desc>
  <concept_significance>100</concept_significance>
 </concept>
 <concept>
  <concept_id>10003033.10003083.10003095</concept_id>
  <concept_desc>Networks~Network reliability</concept_desc>
  <concept_significance>100</concept_significance>
 </concept>
</ccs2012>
\end{CCSXML}

\ccsdesc[500]{Computer systems organization~Embedded systems}
\ccsdesc[300]{Computer systems organization~Redundancy}
\ccsdesc{Computer systems organization~Robotics}
\ccsdesc[100]{Networks~Network reliability}

%%
%% Keywords. The author(s) should pick words that accurately describe
%% the work being presented. Separate the keywords with commas.
\keywords{Smart watch, text entry, touch input.}

%%
%% This command processes the author and affiliation and title
%% information and builds the first part of the formatted document.
\maketitle

\section{introduction}

%文本输入是平板电脑上的刚性需求,用户使用触屏软键盘来完成各种文本输入任务如搜索、办公。然而,触屏软键盘在输入效率(vs)、疲劳程度[xx]和视觉注意力负担[xx]等诸多方面上都比不上物理键盘。用户可以将手指休息在物理键盘上,却不能将手指休息在触屏键盘上,因为触屏键盘上的休息行为会导致误识别。这一点不同给物理键盘带来了两大优势:一是减轻了疲劳,保证了用户高效输入的持续性[xx];二是用户可以通过触摸按键纹理来对齐他们的手指,从而实现盲打,大幅度提高输入效率[xx]。为了弥补物理键盘和软键盘之间的GAP,我们提出了TypeBoard,一款带有防误触功能的触屏软键盘,它允许用户将手指休息在触屏键盘上。TypeBoard的一个直观的好处是减低疲劳。更进一步,在TypeBoard算法的帮助下,我们有机会通过添加膜层[xx]或可变形触屏[xx]等方案给软键盘添加触觉纹理,从而在软键盘上支持盲打,提高输入效率。

Increasingly more people are using tablets for text input tasks such as searching the Internet, writing email messages, and office work \cite{2018-Japanese}. They use software keyboards on the touchscreens of the tablets. However, there is a gap between the tablet keyboard and the physical keyboard in aspects of the fatigue problem \cite{2014-Differences}, switching of visual attention \cite{2017-BlindType, 2010-NoLook, 2010-Eyes}, and typing speed \cite{1991-Improving, 2011-Typing}. Users can rest their fingers on the physical keyboard but cannot rest on the tablet keyboard because touching the screen causes misrecognition. The resting behavior plays an important role in the usability of physical keyboards. First, it reduces fatigue \cite{2013-TapBoard}. Second, users align their fingers by touching the keys and achieve touch typing \cite{2010-Warning, 1995-Use, 2011-Hierarchical, 2015-Haptic}. Touch typists type quickly because they do not use the sense of sight to find the keys. To bridge the gap between tablet keyboard and the physical keyboards, we proposed TypeBoard, a software keyboard that prevents unintentional touch. As figure \ref{fig:teaser} shows, the TypeBoard allows users to rest their fingers on the touchscreen. The TypeBoard prevents unintentional touches such as fingers resting and thenar eminence touches while passing users' typing behaviors. Furthermore, if we provide tactile landmarks on the TypeBoard through deformable screens \cite{Website-Tactus}, and changeable surface texture \cite{2011-Stimtac, 2010-TeslaTouch, 2011-Enhancing}, the TypeBoard can support touch typing.

% (e.g., 38 WPM on iPad vs. 69 WPM on MacBook \cite{2010-Keyboard})
% TypeBoard prevents unintentional touches such as fingers resting and thenar eminence touches while passing users' typing behaviors.

\begin{figure}[!htbp]
	\centering
	\includegraphics[width=\textwidth]{figures/teaser.png}
	\caption{The TypeBoard is a software keyboard that prevent unintentional touch such as fingers resting and thenar eminence touches. }
	\label{fig:teaser}
\end{figure}{}

%Previous work has demonstrated the feasibility of providing tactile landmarks on touchscreens through deformable screens \cite{Website-Tactus} and changeable surface texture \cite{2011-Stimtac, 2010-TeslaTouch, 2011-Enhancing}. Under this premise, the TypeBoard with tactile landmarks is expected to support touch typing on the touchscreen.

%在触屏键盘上区分打字点击和“误触”并不是我们的首创。在2013年,TapBoard就曾经提出将Tapping动作看作打字点击,而将其它触摸事件视为误触。Tapping的定义是“触摸时间低于xx毫秒,位移低于xx毫米的点击”。用户需要主动去适应TapBoard的技术方案,这影响了用户输入的自然性和舒适性,同时这种基于阈值的方法识别准确率也不高(xx.x\%)。为了克服TapBoard的局限性,这份工作从用户的角度出发,将打字点击定义为“表达键入意图的点击”。在此前提下,我们通过用户实验来收集数据,探索用户的打字行为,再根据用户行为设计TypeBoard防误触算法。

We are not the first to explore the detection of unintentional touch on the touchscreen keyboard. In 2013, Kim et al. proposed TapBoard \cite{2013-TapBoard}, a touchscreen keyboard that regards tapping actions as keystrokes and the other touches as unintentional touches. Tapping actions were those touches that neither exceed a timeout threshold of 450 ms nor exceed a displacement threshold of 15 mm. Users adapted to the thresholds, which led to unnatural typing behavior. TapBoard cannot judge the intention of touch until the touch is released. Even so, the accuracy is only 97\%. We have two strategies to overcome TapBoard's limitations of naturalness and recognition performance. First, to ensure natural interaction, we defined \emph{unintentional touches} as those touches that do not intend to input words. Second, we conducted user studies to understand user behavior, based on which we designed the algorithm of TypeBoard.

Techniques change user behavior. For example, users will not rest their fingers on the tablet keyboard unless the touchscreen prevents unintentional touches. Thus, to design TypeBoard, it is valuable to understand the users' typing behavior on the TypeBoard itself. However, how can we observe the user behavior on the TypeBoard before we have developed the TypeBoard? To solve this "chicken and egg" problem, we followed an iterative process, i.e., we developed a semi-finished TypeBoard, then analyze the user behavior on it, and finally improved the technique by using the latest dataset. In details, we conducted three user studies, each contributing to answering one of the three research questions as follows:

%然而,“探索用户在TypeBoard上的打字行为”是困难的,因为用户的行为会受到防误触算法的能力的影响。例如,用户在防误触软键盘上打字时,可能会将手指休息在触屏上,带来更多对识别有挑战性的误触点。这是一个蛋鸡悖论,即我们如何在设计出防误触触屏键盘之前,探索用户在防误触键盘上的打字行为呢?为了解决这一问题,我们采用了迭代的方案。我们共组织了三个用户实验,分别回答了以下三个研究问题:

\begin{enumerate}
	\item{\textbf{RQ1):} \emph{What is the user behavior on an imaginary TypeBoard?} We conducted study one to collect users' typing data on a touchpad that has no feedback. Participants could not input words, instead, they pretended to input words, and imagined that the touchpad prevents unintentional touch. We found eleven categories of unintentional touches. The three most common ones were multiple finger resting, hypothenar eminence touches, and repeated touch reporting. We developed a semi-finished TypeBoard based on the analysis of user behavior.}
	\item{\textbf{RQ2):} \emph{What is the user behavior on the TypeBoard?} We conducted study two to collect users' typing data on the (semi-finished) TypeBoard. We did not found new unintentional touch behavior. However, the user behavior in study two was different from study one in details. For example, the average force of intentional touch was 33.78\% lighter in study two because users learned that they could type without much effort, gradually making the touch lighter. We used the data in study two to improve the TypeBoard, achieving an accuracy of 98.88\%, with a delay of 100 ms.}
	\item{\textbf{RQ3):} \emph{What is the performance of the TypeBoard? What if there are tactile landmarks on the TypeBoard?} We conducted study three to evaluate users' typing performance on three settings, including ordinary tablet keyboard, TypeBoard, and TypeBoard plus (the TypeBoard with tactile landmarks). Users were expected to perform touch typing on the TypeBoard plus. 【结果未知】Results show that there is no significant difference in typing speed between regular keyboard and TypeBoard. Users tend to rest their fingers on the TypeBoard and feel more relaxed than the regular keyboard. TypeBoard plus is significantly the best in aspects of both typing speed and subjective feedback. TypeBoard plus improves the typing speed on touchscreen keyboard by xx.x\%.}
\end{enumerate}

%\begin{enumerate}
%	\item{\textbf{RQ1):} \emph{用户在一个想象中的防误触触屏键盘上的打字行为如何?}我们组织实验一,在无反馈的情况下收集了用户的打字数据,实验中用户需要想象该键盘能够完美地防止误触。由于触摸板没有反馈,用户不能真的打字,而是想象字母上屏了。用户完成了多种文本输入任务,并人工标注了数据。实验共采集了xx个数据点,其中误触点占xx\%。我们分析了用户的行为,并开发了针对性算法,识别误触的准确率达到了xx.x\%,延迟为手指落下后100毫秒。作为比较,TapBoard[xx]在该数据集上的准确率仅为xx.x\%,且只能在手指抬起时识别。}
%	\item{\textbf{RQ2):} \emph{用户在防误触触屏键盘上的打字行为如何?}在实验一以后,我们得到了初版的TypeBoard,我们组织实验二收集了用户在该TypeBoard上打字的数据。在实验二中,用户能够键入字母,同时得到声音反馈,TypeBoard也会忽略用户的误触点。实验二共采集到xx个数据点,其中误触误触点占xx\%。在实验二的实验过程中,TypeBoard的预测准确率为xx.x\%,比实验一中的结果略低。这是由于实验所采集到的用户行为的不同造成的。统计发现,实验二的数据与实验一相比存在着多方面的差异,例如点击力度显著更轻、各类误触行为频次发生变化等等。也就是说,实验二的数据更好地描述了用户在真正防误触触屏键盘上的打字行为。我们使用实验二的数据集重新训练了模型,其准确率提高至xx.x\%。}
%	\item{\textbf{RQ3):} \emph{防误触键盘对用户体验和效率有和影响?防误触键盘上加上纹路有何影响?} 实验三有两个目的,一是对比TypeBoard和普通触屏键盘的性能差异;二是验证在TypeBoard上加上触觉纹理后,是否能支持用户在触碰键盘上的盲打,从而提高输入效率。因此在实验三中,我们通过一个within-subject实验,验证了用户在以下三种设置下的打字效率:一是普通的触屏键盘,二是TypeBoard,三是TypeBoard+tactile landmarks。结果显示,TypeBoard能让用户将手指休息在键盘上,从而主观上感觉疲劳被减轻。而TypeBoard+tactile landmarks的效果是最好的,不仅能让用户更多得将手休息在键盘上,而且能让用户通过纹理来对齐手指,显著地提高了文本输入的效率。}
%\end{enumerate}

The contribution of this work is three-fold. First, we proposed TypeBoard, which identifies unintentional touch in text input tasks with high accuracy (98.88\%) and low latency (100 ms). Second, we explored the user behavior on the touchscreen keyboard with unintentional touch prevention. We published the dataset. Third, 【结果未知】we empirically demonstrated that TypeBoard could reduce fatigue, while TypeBoard plus tactile landmarks significantly improve touchscreen keyboard in both typing speed and user experience.

%这份工作有三个贡献点:第一,TypeBoard准确地、低延迟地区分了触屏打字时的typing和误触。第二,在TypeBoard的支持下,我们理解并总结了用户在防误触触屏键盘上的输入行为,我们公开了该数据集。第三,我们通过评测实验证明,TypeBoard(特别是加入tactile landmards的情况)与传统触屏键盘相比,提高了输入效率,降低了用户疲劳程度。


\section{Related Work}

1. 触屏上的防误触算法

(1)按误触的定义分类

我们可以通过对误触的定义来分类相关工作种触屏上的防误触算法。有相当一部分工作将误触笼统地定义为“无意的点击”[xx,xx,xx],有的工作则将误触一词描述为xx[xx]和xx[xx]。这一类的工作选择有代表性的具体描述一下。这一类工作的优点在于泛化能力强,(声称)能适用于各种各样的场景和任务。其缺点是准确率普遍较低,几乎所有工作的预测准确率都低于95\%[xx],这一低准确率可能暗示“有意的点击”和“无意的点击”这一笼统的分类方法有时候是不可区分的,而不会像future-work所说的一样随着机器学习的进步而解决。另外,许多工作并没有在多个应用场景下测试该防误触算法[xx],这使得泛化能力面临质疑。

[2020-TabletopTouch]
In our research, we leverage gaze direction, head orientation and screen contact data to identify and filter out unintentional touches. 强调是大桌面. They defined unintentional touches as those touches that do not contribute to any interaction goal. They compiled our findings into five types of spatiotemporal features, and train a machine learning model to recognize unintentional touches with an F1 score of 91.3\%.

[2012-IdentifyUnint]
手持触屏设备上(电容屏)的防误触算法,对有意点击的recall-rate是99.2\%,但precision只有79.6\%。能用的信息只有报点(并没有电容图),仅用到了简单的时空特征,包括触摸时间,触摸移动距离,与边缘的距离,轨迹距离与直线距离的比例。Matero and Colley [36] tried to classify accidental touch events on mobile phones in daily usage. They investigated numerous typical operations on mobile devices (e.g., sweeping on the screen, device handling and phone calls) and proposed a rule-based classification algorithm according to the contact area and touch duration. Their best algorithm can eliminate 79.6\% of unintentional touches with a 0.8\% false-positive rate.

另一部分工作限定了场景和任务,因此对误触的定义也更加具体一些。例如,有的工作专门处理文本输入法中的误触问题[xx],这一类工作中的误触就是非有意的打字点击;又例如,有的工作专门处理触屏笔使用过程中的误触问题[xx]\cite{2014-PalmRejection},这一类工作的误触就是非触摸笔输入。这一类工作选择有代表性的具体描述以下。这一类工作的缺点是泛化能力弱,仅限于一类应用场景和任务。其优点是准确率能够达到很高。由于场景单一、简单,这一类工作得以分析fail-cases的原因,从而探索用户的行为模式,我们认为这一点在人机交互和机器学习融合的领域里是十分重要的。

[2014-PalmRejection]
Schwarz et al. present a probabilistic touch filtering approach that distinguish between legitimate stylus and palm touches. They used spatiotemporal features and decision forest model to distinguish palm touches from stylus input. Their system improves upon previous approaches, reducing accidental palm inputs to 0.016 per pen
stroke, while correctly passing 97.9\% of stylus inputs.


(2)按输入信号分类

我们还可以根据输入的信号来分类触屏上的防误触算法。大部分工作使用传统的触屏设备中的报点数据[xx]或者电容屏图像[xx]来判断误触。这一类的工作选择有代表性的具体描述一下。这一类工作的优点在于仅利用了电容屏数据,可以应用在目前大部分手机和平板电脑上;缺点在于其准确率不够高,如果能结合其它传感器数据,可能可以提高识别精度。

另一部分工作在正规的电容屏之上加上了其它的输入信道,比如压力[xx]、声音[xx]等信息,甚至是结合了四肢[xx]、肢体语言、头动眼动[xx]等触屏之外的信息。这一类工作选择有代表性的具体描述一下。这一类工作的优点是识别准确率更高,缺点是不能马上应用在现有的已经普及的设备上。

在我们的这份工作中,我们将任务focus在文本输入任务上,在输入信道上,我们也引入了压力屏的压力数据,以增强识别的准确率,希望把这个防误触的问题研究透彻,也把防误触的算法做到完美。有一份叫TapBoard的工作和这篇论文很像[xx],但它有不少缺陷。例如xx,xx,xx。另外,TapBoard在实验设计上有一个错误,他们通过一个基于誊写任务的对比实验,证明了TapBoard的输入效率和传统触屏键盘无异,这是以偏概全的。我们通过Pilot-study发现,在誊写等快速打字的任务中,用户很少将手指休息在触屏上,因此TapBoard无法证明用户在写日记等任务中不会因为手指的休息造成误触问题。

我们这份工作和先前工作相比的一个特点是,我们首次考虑到了防误触能力和用户行为之间互相影响的效应。有的实验设置在采集数据时不给出反馈[xx],有的实验是给出了未经防误触算法过滤的反馈[xx],这都不能正确反映用户在最终技术上的心理模型。在这份工作中,我们采用迭代式的方法,重复地采集用户数据和优化机器学习算法,最终总结出用户在一个防误触触屏键盘上打字的行为,并根据此用户行为,设计了一款鲁棒的防误触触屏键盘。

2. 触屏键盘上支持防误触的好处

(1) 有机会弥补触屏键盘上没有触摸状态的问题

物理键盘上有release/touch/press三态,而触屏键盘上只有release/press两态[xx]。touch状态在物理键盘上承担着重要的作用,具体有三点:1、物理键盘上的触觉纹理可以让用户不看键盘的情况下对齐手指,从而支持盲打[xx];2、键帽能够给用户提供触觉反馈,确认一次点击操作,这不仅增强了用户体验[xx],也显著地提升了打字速度[xx];3、在物理键盘上,用户可以将手指休息在键位上,降低了疲劳程度[xx]。触屏键盘上touch状态的缺少,使得触屏上的打字效率显著低于物理键盘打字的效率。

有不少工作尝试弥补物理键盘和触屏键盘之间的gap。xx通过加装xx硬件,提供了纹理信息,达到了xx效果。xx通过加装xx硬件,提供了用户在打字时的触觉反馈,达到了xx的效果。xx采用简单的时间和距离阈值方法来区分打字和误触,从而允许用户将手指休息在触屏键盘上。然而,以上对触屏键盘的体验提升,都是以能够正确区分打字和误触为前提的。例如,对于屏幕上有tactile反馈的工作中[xx],只有在触屏键盘能够很好地防误触地情况下,用户才会去摸键盘上的纹理,从而使能触屏上的盲打。

(2) 在touchscreen上无缝区分keyboard和其它有意输入

如果能够在touchsreen上identify-typing-action,就可以仅使用触摸屏的同一块区域,同时支持文本输入和其它交互,如触摸屏控制鼠标pointing[xx],又如其它的一些手势命令[xx]等等。TapBoard2能够有效地将打字的触摸事件和pointing的触摸事件正确区分,使得同一块触摸屏及支持了文本输入,又支持了触摸板控制鼠标。xx将手掌的误触作为输入通道。

还可以区分typing和stylus输入

[2014-PenUnint]

[2014-PenMightier]

[2014-PalmRejection]

[Exploring and Understanding Unintended Touch during Direct Pen Interaction]

(3) 将原本应该是误触的触碰利用为输入信息

有的相关工作将误触作为有用的输入信息,从而提供额外的输入通道。

Matulic et al. [2017-HandContact] extended hand interactions from fingertips to the whole hand in hand-shape based interaction. Tabletop interaction can be enriched by considering whole hands as input instead of only fingertips. We describe a
generalised, reproducible computer vision algorithm to recognise hand contact shapes, with support for arm rejection, as well as dynamic properties like finger movement
and hover. A controlled experiment shows the algorithm can detect seven different contact shapes (such as fist, flat palm and spread hand) with roughly 91\% average accuracy如果能将type点击事件和手掌的触摸事件区分开来,就可以支持hand contact shape所支持的交互方式。

Zhang et al. [77] proposed to leverage various hand postures such as using the palm to augment pen and touch interactions.

[2018-PalmTouch]

将大鱼际的误触作为模式切换。

We present PalmTouch, an additional input modality that differentiates between touches of fingers and the palm. We present different use cases for PalmTouch, including the
use as a shortcut and for improving reachability.  We have developed a model that differentiates between finger and palm touch with an accuracy of 99.53\% in realistic scenarios.

[??] 比如处理漂移问题。




\section{Study 1: User Behavior on Imaginary TypeBoard}

在这个实验中,我们采集了用户在一个没有反馈的触屏键盘上的打字数据,实验中我们要求用户想象该键盘可以防误触。本实验的目标是研究用户在想象中可以防误触的键盘上打字的行为,从而指导触屏键盘防误触算法的设计。用户在填写个人信息、描述个人爱好、模拟开卷考试、看图写话和誊写这五个不同的文本输入任务下打字。由于(不正确的)反馈可能会严重影响用户的输入行为,在实验一的阶段我们要求用户在无反馈的键盘上打字,用户不能真的看到字母上屏,他们需要想象字母能够上屏。实验共采集了xx个数据点,其中xx.x\%是误触点,在该数据集上,我们开发了基于机器学习的防误触算法,Leave-one-out准确率达到了xx.x\%,相比之下,baseline[xx]的准确率仅为xx.x\%。这一结果说明两点:首先,用户在想象中能防误触的键盘上打字时,会引发很多难以仅通过阈值方法区分有意与否的触摸点,对防误触算法带来挑战;第二,结合压力触摸屏上众多的传感信息、结合时间、空间信息,能够大幅度提高触屏上误触点的识别。

\subsection{Design and Procedure}

我们从校园中邀请了16名用户被试,年龄从xx到xx不等,平均数xx,标准差xx,xx名女性。所有的被试都是右撇子,所有用户有超过xx年的手机文本输入经历,xx名用户常用平板电脑进行文本输入。图xx展示了实验的设置,桌子上放置着一块morph-sensel压力触摸板和显示器,用户可以根据自己的需求调整椅子的高度、压力触摸板和显示器的位置。压力触摸板上没有Qwerty的布局,这意味着用户可以需要根据自己盲打的经验,大概地点中每个字母在触屏上地位置。这一设计是为了减轻用户将注意力专注在触屏上对用户行为的影响(更接近物理键盘上的盲打行为)。

标注:Annotation

实验分为5个session,分别收集了用户在填写个人信息、描述个人爱好、模拟开卷考试、看图写话和誊写这五种不同的文本输入任务下的打字数据。在输入任务中,由于触屏没有任何反馈,用户不能真的看到字母上屏,而是想象字母上屏了。用户是通过填写一个word文档来完成这五个文本输入任务的,图xx展示了这个word任务文档中每种任务的例子,每个任务的具体描述如下。

\begin{enumerate}
	\item{\textbf{填写个人信息任务:}word文档中有一个表格,其中包含用户姓名、性别、专业等问题。用户操作触摸板(想象中的TypeBoard键盘)和鼠标填写表格。为了保护用户的隐私,用户可以填写错误的信息,前提是他能记住他填写的内容,以便标注时作为参考。用户需要通过鼠标将输入指针移动到表格相应的位置下,然后再触摸板上敲击“输入”,想象所需字母被填写在了word文档中。该任务模拟了真实场景中,需要频繁切换键鼠操作的输入类型。}
	\item{\textbf{描述个人爱好任务:}word文档中有一个问卷,其中包含“最喜欢的城市”、“最喜爱的食物”等个人喜好相关的问题。和任务一相同,用户通过键鼠配合操作完成问卷的填写。该任务模拟了真实场景中简单的问卷填写任务[xx]。}
	\item{\textbf{模拟开卷考试任务:}word文档中包含若干有一定难度的知识性问题,如“比利时的首都在哪里”、“元素周期表中第50号元素时?”。在此任务中,用户很可能不知道问题的正确答案,此时她需要通过搜索引擎来得到答案。用户在使用搜索引擎时存在一个问题是,她输入的文字并不能真的上屏,此时我们要求用户边输入边把搜索的关键词说出来,实验者帮忙使用键盘完成搜索。both搜索的过程和填写试卷的过程是本实验采集的数据。该任务模拟了真实场景中常见的搜索引擎任务。}
	\item{\textbf{看图写话任务:}word文档中包含了一张简笔画(如图xx的xx所示),用户需要根据这幅图写一个五句话的小作文。在此任务中,我们同样要求用户边说边写,实验者在一旁进行速记,这是为了在标注中给用户提高上下文。该任务模拟了真实场景中常见的创作型文本输入[xx],这一类任务有着xx的特点。}
	\item{\textbf{誊写任务:}word文档中包含了一句话,用户需要尽可能快地将这句话誊写五次。每个用户所誊写的句子都是不同的,都是从phrase-set[xx]中随机抽取。誊写任务是文本输入工作中最常见的评测任务,适用于评测输入法的输入效率上限。}
\end{enumerate}

在每个session结束后,用户需要通过一个交互式程序来标注刚刚所进行session的报点数据,该程序同时展示了实验过程中的录屏视频和压力板图像,用户需要结合压力板图像和录屏视频提供的上下文信息,给压力板图像中的每个报点标注。如图xx所示,压力图像中所包含的报点初始默认值为负例(误触),用红色点表示。用户若认为一个报点是正例(打字事件),则需要用鼠标左键将该点标为绿色;用户可以通过鼠标右键将报点重新标为红色;用户若不清楚该报点是正例还是负例,则需要用鼠标中键将该点标为蓝色,表示剔除出数据集。本实验的输入任务部分,五个session加起来大约需要15分钟,标注过程大约45分钟,每两个session之间强制休息5分钟时间来避免疲劳,实验总时长为80分钟。

\subsection{Apparatus}

Morph Sensel简介,包含多少像素的压力数据,压力数据的范围是xx,精度是xx。自带报点功能,多大的触摸点、多重的点击算是一个报点,压力的报点阈值比较低,不会出现漏报正例的情况。

电脑设置,运行速度-->程序运行帧率是稳定的50FPS。

\subsection{Result}

原始数据共包含xx个数据点,其中用户无法区分的点的占比为xx.x\%,刨去这些数据点之后正例(打字事件)的比例是xx.x\%,负例(误触点)的比例是xx.x\%。我们先使用简单的机器学习方法对数据进行验证,当机器学习结果和用户标注结果有大量不同时,我们会将用户召回,让她重新标注这些分类错误的数据,其中,有xx.x\%的数据的确是用户标错了(有的用户对误触的理解出现了偏差),剩下xx.x\%的数据是机器学习预测错误,其中xx.x\%的误触和xx.x\%的漏报。我们对这些简单的机器学习不能正确分类的数据点进行了人工观察,总结出来这些无触点的分类如下表所示。

【表格:简单机器学习容易分类出错的误触点类型】

从表格xx能够看出来,大多数误触点都是有明显的特点的,我们针对几乎每种错判,都设计了专门的特征向量,用于机器学习。只有表中加粗的两类错判很难找到特征,这一类数据的占比为xx.x\%,用户仅能通过任务的上下文来标注,而系统不可能准确知道用户想要输入的文字,因此我们认为这一部分数据的分类问题是不可解的。因此在该数据集上,机器学习的准确率上限(贝叶斯概率)大概在xx.x\%,这也是我们recognition的目标。

我们采用支持向量机来实现二分类模型,特征向量如表格xx中间的一列所示,每一段特征向量独立的预测准确率也公布在表格中了。将所有特征结合起来的准确率为xx.x\%。进一步的,由于空间相邻触点特征组、时间相邻触点特征组和触点位置和位移特征组之间比较独立,我们用触摸点自身特征组分别结合上述三组特征,训练了三个二分类模型,再用投票算法[xx]将三个模型融合起来,准确率进一步提高至xx.x\%,接近贝叶斯概率。

\subsection{Discussion}

误触点的构成:休息、手掌误触、输入时误触的比例。

实验任务显著影响了用户行为。

初版TypeBoard算法的实际使用体验(有反馈)。存在问题:数据集还不够全面,在有反馈的使用场景下,存在一些可复现的误触。因此,我们有必要设计实验二,来调查用户在有反馈的触屏键盘上的打字规律。



\section{Study 2: User Behavior on the TypeBoard}

In this study, we obtained users' typing data on the semi-finished TypeBoard, the touchscreen keyboard developed in study one that prevents unintentional touch. The motivation was to investigate users' typing behavior, based on which we can further improve the TypeBoard.

%we recruited 16 participants and divided them into four groups equally. The first group of participants typed on the existing TypeBoard. After each group finished the experiment, we used the data to improve the TypeBoard. The following participants typed on the improved TypeBoard. That is, we explored the user behavior and improve the method through an iterative process.

\subsection{Participants}

We recruited 16 participants from the local campus (aged from 19 to 37, M = 22.19, SD = 4.55, six females). All the participants were right-handed and did not take part in the first study. They have used software keyboards on smartphones for not less than one year (M = 5.88, SD = 2.36). Ten participants have ever used software keyboards on tablets.

\subsection{Design and Procedure}

As figure \ref{fig:study2_illu} shows, the experimental devices were the same as those in study one, including a Windows Surface tablet computer and a Sensel Morph pressure-sensitive touchpad. The working frequency was 50 FPS. There were text entry tasks, which were the same as study one, namely filling in personal information, short questions, open-book examination, and picture writing. The differences in study two are as follows. First, participants could input words. Second, the keyboard prevented unintentional touch. Intentional touches would trigger keystrokes and audio feedback, while the system ignored unintentional touches.

\begin{figure}[!tbh]
	\includegraphics[width=0.9\linewidth]{figures/study2_illu.png}
	\centering
	\caption{The experimental setting of study two.}
	\label{fig:study2_illu}
\end{figure}

Before the experiment, participants warmed up through five-minute free writing. Because no participant had experience typing on a keyboard with unintentional touch prevention, we reminded the participants that they could rest their fingers on the keyboard. Participants could decide to rest their fingers or not according to the task and their preference.
After finishing each task, the participant labeled the data using the label program introduced in study one. During the labeling process, we compared participants' labels with the predictions by the semi-finished TypeBoard. When the two results were different, we recorded the data point for later processing. If the experimenter could not explain the difference, he would discuss it with the participant.
On average, participants spent 10 minutes completing the text entry tasks and spent one hour labeling the data. Participants rested for five minutes between two tasks to avoid fatigue. The study lasted for 90 minutes. The experiment took more time than study one because participants needed to input correct words in this study.

%In a few cases, the system repeatedly gave an incorrect response. For example, a participant (P8) usually rested her left thumb on the screen and triggered false touch. This behavior was not observed in study one, so the system could not identify it correctly. In this situation, we encouraged the participant to keep their behaviors when the system gave an incorrect response. In this situation, participants needed to imagine that the desired words were corrected entered.

%Because the ability of unintentional touch rejection will affect the user behavior, we used an iterative process to improve TypeBoard, and ensured that the participants were typing on the latest version of TypeBoard. We divide the 16 participant into four groups. The first group of participants complete the experiment on the naive TypeBoard introduced in study one. After each group of participants finish their experiment, we improved the algorithm by updating training data and adding features in model. We expected that the TypeBoard algorithm was getting stronger, while the data we collected was getting closer to the natural behaviors of typing on a perfect keyboard.

\subsection{Result}

%在实验的过程中,每四个用户的实验结束之后,我们都会采用新的数据来优化防误触算法,如图xx所示是防误触算法随着完成实验人数的增加的变化,其中每个测试点的数据集包含实验二过程中已经收集到的数据加上实验一的所有数据,评测方法是leave-one-out检测。结果表明,随着实验人数的增加,迭代的TypeBoard和初版TypeBoard的差距在显著拉大。这说明TypeBoard的防误触能力在增强,我们采集到的数据也越来越接近用户在一个完美防误触键盘上的打字行为。【图:初版TypeBoard、迭代TypeBoard随着实验人数增加,准确率的变化】

The dataset contained 13789 touches, excluding the ambiguous touches (0.22\%) in the labeling process. After the double-check of labels, the dataset consisted of 71.01\% positive samples (intentional touches) and 28.99\% negative samples (unintentional touches). The semi-finished TypeBoard predicted the intention of touch with an accuracy of 98.05\% (SD=1.51\%) in study two. There were 0.39\% (SD=0.37\%) false positives and 1.56\% (SD=1.37) false negatives. That is, participants encountered 2.20 unrecognized touchpoints and 0.55 false triggering touchpoints every 100 keystrokes in study two.
The recognition rate of the semi-finished TypeBoard dropped from 99.07\% on the study one dataset to 98.05\% on the study two dataset. It indicates that the user behavior changed in study two, so the model trained on the study one dataset did not work perfectly in study two. Because the user behavior observed in study two was closer to the user’s real usage, it is valuable to investigate the differences in user behaviors between the two studies.

% 实验二共收集了13789个数据点,已经排除了0.22\%的用户无法区分的数据点。在经过了和实验一相同的label确认流程之后,数据集中共包含71.01\%的正例和28.99\%的负例。实验一中总结出来的模型在实验二的测试集下,预测准确率为98.05\%(SD=1.51\%),有0.39\%(SD=0.37\%)的漏报和1.56\%(SD=1.37)的误触发。也就是说,在实验二的实验过程中,用户每有100次有意点击中,平均有0.55次漏报和2.20次的误触点。

Compared with study one, we did not find new cases of unintentional touches in this study. However, the user behavior in study two was different from the last study in details. Table \ref{tab:behavior_difference} shows the examples. First, the user's average pressure of intentional touches in this study was significantly lighter than the last study ($t_{15}=2.78, p<.01$). When typing with feedback, users found that they could type letters without much effort, gradually making the touch lighter. Second, there were more continuous touches in this study ($t_{15}=3.57, p<.005$), where a touch is close to the last touch in both time intervals (< 500 ms) and distance (< 0.5 key widths). This is because participants need to remove incorrect words by continuously pressing the backspace in real typing tasks. Third, the study two dataset contained fewer rollover-typing than the study one ($t_{15}=-6.32, p<.001$). Rollover-typing means the next key is pressed before the previous is released. Participants typed slower in study two because they needed to enter correct words. This slower typing speed correlated with fewer keystrokes typed with rollover \cite{2018-Observations}.

% 显著性分析是否支持该结论,用户点击的力度随着任务顺序而递减?

%Besides, some participants used the right key to select the desired work in the candidate list.

% Table generated by Excel2LaTeX from sheet 'Sheet2'
\begin{table}[htbp]
  \centering
  \caption{The differences of user behavior between the two studies. We use t test to evaluate the significance of the difference. If Levene's test rejects the homoscedasticity of data, we use unequal variances t test instead.}
    \begin{tabular}{|p{5em}|p{15em}|p{4.8em}|p{4.8em}|p{3.5em}|p{4.8em}|}
    \toprule
    \textbf{Measure} & \textbf{Introduction} & \textbf{Study one} & \textbf{Study two} & \textbf{Levene's test} & \textbf{T test} \\
    \toprule
    Touch pressure & The average touch pressure of intentional touches in grams. & 188.39g (SD=64.72) & 124.75g (SD=60.71) & Reject & $t_{15}=2.78$, $p<.01$ \\
    \midrule
    Continous touches & The continous touches as a percentage of all intentional touches. & 4.02\% (SD=1.96\%) & 11.89\% (SD=4.41\%) & Pass  & $t_{15}=3.57$, $p<.005$ \\
    \midrule
    Rollover-Typing & The rollover-typing touches as a percentage of all intentional touches. & 17.60\% (SD=9.34\%) & 7.73\% (SD=5.22\%) & Pass  & $t_{15}=-6.32$, $p<.001$ \\
    \toprule
    \end{tabular}%
  \label{tab:behavior_difference}%
\end{table}%

%\midrule
%Multiple fingers resting & The touches caused by multiple finger resting as a percentage of all unintentional touches. & \multicolumn{1}{r|}{} & \multicolumn{1}{r|}{} & \multicolumn{1}{r|}{} & \multicolumn{1}{r|}{} \\

% The system will classify user touches as either intentional (positive) or unintentional (negative).

As figure \ref{fig:unintentional_touch} shows, we counted the unintentional touches caused by each case of user behavior. The number was counted in touches, e.g., we counted the five fingers resting behavior five times. The three most common unintentional touches were multiple fingers resting (82.90\%), hypothenar eminence touching (7.53\%), and extra light touchpoint (3.21\%). For the unintentional touches illustrated with translucent sub-figures, including (h) one finger resting and (i) extra heavy touchpoint, humans could not identify their intention without contextual information. The machine could hardly judge these cases. Fortunately, their proportion in all unintentional touches was only 1.52\%.
We retrained the model using the study two dataset. Leave one out cross-validation shows that the accuracy increased to 98.88\% (SD=0.73\%), significantly surpassed the semi-finish TypeBoard (F = xx, p < xx). There were 0.66\% (SD=0.58\%) false positives and 0.46\% (SD=0.45\%) false negatives. On average, TypeBoard users will encounter 0.65 unrecognized touchpoints and 0.93 false triggering touchpoints every 100 keystrokes. So far, we have finished the design of the TypeBoard.

\begin{figure}[!tbh]
	\includegraphics[width=1.0\linewidth]{figures/unintentional_touch.png}
	\centering
	\caption{The classification of unintentional touches. The percentage in the bracket is the proportion of the corresponding cases in all unintentional touches.}
	\label{fig:unintentional_touch}
\end{figure}

%在算法方面,我们也需要将预测算法的训练集替换成实验二所采集到的数据集,重新训练机器学习模型。果然,Leave-one-out验证发现算法的识别准确率提高至98.88\%(SD=0.73\%),且这一提高是显著的(F,p)。

\subsection{Discussion}

\subsubsection{Comparison to previous work}

TapBoard \cite{2013-TapBoard} was the latest study that similar to our work. 
Please note that the TapBoard is different from our proposal TypeBoard. We both investigate unintentional touch prevention on touchscreen keyboards. However, the TapBoard has two defects. First, the TapBoard recognizes tapping actions as keystrokes and the others as unintentional touches. Tapping actions were those the duration is shorter than 450 ms, and the displacement is within 15 ms. Users adapt their behavior to meet the technique, which is not natural and relaxing. Second, the TapBoard cannot judge a touch until the touch is released. Even so, the recognition rate is only 97\%.
%Third, the TapBoard's evaluation was conducted on transcription tasks. The experimental task is not challenging because users seldom rest their fingers on the touchscreen in the transcription task, which was proved in our study three.

We evaluated the TapBoard on our study two dataset, where participants performed natural typing behaviors on the touchscreen keyboard that prevents unintentional touch. The recognition accuracy of the TapBoard was xx.x\% on our dataset, which is nowhere near the performance of our proposal (98.88\%). The result shows that our proposal outperforms existing work.

\subsubsection{The user differences}

As figure \ref{fig:u_touches_over_user_task}(left) shows, the user behaviors of different participants vary a lot. For extreme examples, P1 generated 74.40 unintentional touches every 100 keystrokes, while P16 only produced three unintentional touches during the experiment. We divided participants into three categories. The first type of users (P1-P10) were willing to rest their fingers on the touchscreen. They generated a lot of unintentional touches during the experiment. The second type of users (P11-P12) believed that TypeBoard could prevent unintentional touches. They put their palms on the touchpad but seldom rested their fingers on the keyboard. P12's comment was the key: "\emph{I do not rest my fingers on the touchscreen because I cannot press down directly in the touched state.}" The third type of users (P13-P16) hanged the wrist to input. Surprisingly, they varied a lot in touchscreen keyboard expertise. P13 and P14 had only used touchscreen keyboards for one year, P15 had no experience, while P16 was an expert who performed touch typing on touchscreen keyboards. Results show that personalization is promising to improve the TypeBoard.

\begin{figure}[!tbh]
	\includegraphics[width=1.0\linewidth]{figures/u_touches_over_user_task.png}
	\centering
	\caption{The unintentional touches per 100 keystrokes over participants and tasks. The blue bars represent those touches that caused by multiple fingers resting, while the orange bars represent other unintentional touches.}
	\label{fig:u_touches_over_user_task}
\end{figure}

\subsubsection{The variety of tasks.}

As figure \ref{fig:u_touches_over_user_task}(right) shows, the task had a significant effect on the number of unintentional touches (F,p). In the four experimental tasks, users produced xx.x (SD=xx.x), xx.x (SD=xx.x), xx.x (SD=xx.x), and xx.x (SD=xx.x) unintentional touches every 100 keystrokes. Bonferroni-corrected post hoc tests showed significant differences between the following task pairs: 1-3 (p<), 1-4 (p<), 2-3 (p<), 2-4 (p<). Users generated more unintentional touches in the first two tasks, which involved more frequent switching between text input and cursor control. Results show that the variety of tasks is important for exploring unintentional touch but usually being neglected in studies \cite{2013-TapBoard, 2020-TabletopTouch, 2014-PalmRejection}.

%The percentages of unintentional touches among the four tasks were 29.73\% (SD=23.71\%), 29.18\%(SD=22.06\%), 18.02\%(18.88\%) and 17.62\%(SD=13.76\%). Bonferroni-corrected post hoc tests showed significant differences between the following task pairs: 1-3 (p<), 1-4 (p<), 2-3 (p<), 2-4 (p<). This result indicates that users generated more unintentional touches on the task with more frequent switching between pointing and typing. We argue that the variety of tasks is important but usually neglected in unintentional touch studies.

\subsection{Why sample five frames in each touch?} 

The more frames we sample in each touch, the more accurate the prediction is. However, a long sampling window means a large delay (20 ms per frame), which affects the user experience. There is a trade-off between the delay and the recognition accuracy. To strike a balance, we simulated our algorithm with different delays. Figure \ref{fig:error_rate_diss}(left) shows the results. An RM ANOVA showed that delay had a significant effect on recognition accuracy ($F_{4,56}=4.88, p<0.05$). Pair-wise comparisons showed significant differences between the following delay pairs: 60-100 (p<.05), 60-120 (p<.05), 60-140 (p<.05), 80-140 (p<.05). Thus, sampling five frames (100 ms) was the best choice, which resulted in acceptable prediction accuracy (98.88\%). Meanwhile, 100 ms is the upper limit of acceptable latency in the touching task \cite{2017-System, 2014-Towards, 2016-Latency}.

\begin{figure}[!tbh]
	\includegraphics[width=1.0\linewidth]{figures/error_rate_diss.png}
	\centering
	\caption{The recognition error rates over delays and sensor abilities. Error bars indicate standard deviation.}
	\label{fig:error_rate_diss}
\end{figure}

\subsubsection{Can our model work with fewer sensors?}

The pressure signal on the touchscreen is helpful for unintentional touch identification. However, most touchscreen devices have no pressure sensor yet, while a few devices have four pressure sensors in the corners (e.g., the force touch trackpad on MacBook). To explored the feasibility of preventing unintentional touch on existing devices, we evaluated the TypeBoard in three sensor settings. 

\begin{enumerate}
	\item{\textbf{Capactive touchscreen:} The commonly used touchscreen devices have capacitive signals but not pressure signals. To evaluate our method on these devices, we removed all the features referred to pressure signals and retrained the model.}
	\item{\textbf{Force Touch trackpad:} The MacBook's Force Touch trackpad has four pressure sensors in the corners. The trackpad provides the total pressure on the whole touchpad. We evaluated our model in this setting by a simulation, where we estimated the pressure of each touch as the product of the total touchscreen pressure and the contact area proportion of the touch.}
	\item{\textbf{Pressure-sensitive touchscreen:} Future touchscreen devices may provide high-resolution pressure signals, which is the experimental setting in our paper.}
\end{enumerate}

%\item{\textbf{Four pressure signals:} Supposed that we can acquire the four pressure signals in the corners, we are able to calculate the pressure of each touch if there are no more than four fingers on the touchscreen. If there are more than four fingers on the touchscreen, we estimate the pressure as we do in the last condition.}

%现实中大多数触摸屏设备还没有压力传感器,有的设备(如MacBook的force touch trackpad)则在四个角上有压力传感器。为了给防误触触屏提供硬件设计指导,我们对比了我们的算法在三种情况下识别准确率,三种情况分别是:(1)capactive-only:只有电容信号;我们将算法中涉及到压力信号的特征维度都阉割掉来训练机器学习模型。(2)four-force-sensor:四个角上有压力传感器,根据实验采集到的高精度压力信号,我们可以根据基础的物理知识模拟四角压力传感器的数据。我们在情况1的基础上,直接加上四个角压力信号的时域特征。(3)force-enabled:既有电容信号,又有压力信号,这也就是我们论文中所述的配置。

Figure \ref{fig:error_rate_diss}(right) shows our method's performances in the three sensor settings. An RM ANOVA shows a significant effect of setting on the recognition accuracy ($F_{2,28}=10.52, p<0.001$). Bonferroni-corrected post hoc tests showed significant differences between the following setting pairs: 1-2 (p<.005), 1-3 (p<.005). The difference between setting two and three was a tendency (p=.062). Results show that the touchscreen devices with the total pressure signal strike a balance between recognition rate and hardware cost.

%Error rate (SD) / False Positive (SD) / False Negative (SD)
%(1) 0.02028 0.01537 0.01262 0.01212 0.00766 0.00619
%(2) 0.01392 0.01158 0.00876 0.01016 0.00516 0.00542
%(3) 0.01269 0.01175	0.0087 0.01055 0.00399 0.00396
%(4) 0.01115 0.00726 0.00659 0.00578 0.00456 0.00451


\section{Study 3: Evaluation on TypeBoard}

The motivations of study three were two-fold. First, we compared the performance and user experience of the TypeBoard and the ordinary touchscreen keyboard. Second, as we introduced in related work, tactile landmarks on keyboards improve users' typing speed by enabling touch typing, so we investigated the feasibility of TypeBoard plus tactile landmarks in this study. In summary, we evaluated users' typing performance on three settings: (1) ordinary software keyboard, (2) TypeBoard, and (3) TypeBoard plus, which was the TypeBoard plus tactile landmarks.

%实验三的目的有两点:第一,评测TypeBoard的性能,包括在不同的文本输入任务下的输入效率和主观用户体验,baseline是没有防误触算法的触屏键盘。第二,相关工作中我们提到,在防误触触屏键盘上,有望通过加上纹理反馈帮助用户盲打,提高输入效率,本实验探索了这一假设是否成立。

\subsection{Participants}

We recruited 15 participants from the campus (aged from 19 to 26, M = 20.87, SD = 2.42, seven females). All the participants were right-handed and did not take part in the previous studies. They have used software keyboards on smartphones for not less than two years (M = 6.67, SD = 2.06). Eleven participants have ever used software keyboards on tablets.

\subsection{Design and Procedure}

The study followed a within-subject design to compare users' typing speed in three keyboard configurations. The participant sat on an office chair. He could adjust the chair to a comfortable position. The participant typed on the pressure-sensitive touchpad to input words and received visual feedback from the tablet. As figure \ref{fig:study3_illu} shows, there were three settings of the pressure-sensitive touchpad in the experiment as follows:

\begin{enumerate}
	\item{\textbf{Config. 1):} \emph{Ordinary Keyboard.} On the ordinary software keyboard, all contacts on the touchscreen are recognized as keystrokes. Users need to hang their wrists in the air to avoid unintentional touches.}
	\item{\textbf{Config. 2):} \emph{TypeBoard.} TypeBoard is a software keyboard with unintentional touch rejection. The system only recognized intentional touches as keystrokes. Users can rest their hands on the keyboard.}
	\item{\textbf{Config. 3):} \emph{TypeBoard plus.} TypeBoard plus refers to the TypeBoard plus tactile landmarks. To provide tactile landmarks on TypeBoard, we attached 0.05 mm thick stickers on the touchpad to simulate physical keys. There were small bumps on the F and J keys, which is the same as the physical keyboard. Users could align their fingers without visual attention.}
\end{enumerate}

\begin{figure}[!tbh]
	\includegraphics[width=1.0\linewidth]{figures/study3_illu.png}
	\centering
	\caption{图:实验三要对比的三种实验设置,大图是用户实验的整体环境,大图中包含三个小图显示三种键盘配置}
	\label{fig:study3_illu}
\end{figure}

There were five sessions for each of the three keyboard configurations. In each session, participants transcribed a Chinese paragraph in a Microsoft Word document. There were roughly xx Chinese characters in a task paragraph. We randomly selected the task paragraphs in a typing speed measurement website \cite{Website-Typing}. Participants were asked to input as fast and accurately as possible. The transcription task is widely used in text entry researches \cite{2003-Metrics, 2003-Phrase, 2017-Word} to evaluate the ceiling typing speed.  We counterbalanced the order of keyboard configuration using a balanced latin square.
Participants had five minutes to warm up before they used each keyboard. They transcribed a paragraph to get familiar with the keyboard. The task phrases in the training step would not appear in the formal experiment. Participants rested for five minutes between sessions to avoid fatigue. On average, participants spent 90 minutes completing the experiment.

\subsection{Reslut}

A Repeated Measures (RM) ANOVA was conducted for text entry speed, Uncorrected Error Rate (UER), and Corrected Error Rate (CER). The within factors were the keyboard and the session. As UER and CER violated the normalcy, we used the Aligned Rank Transform \cite{2011-Aligned} for correction. If any independent variable had significant effects (p < 0.05), we used Bonferroni-corrected post hoc tests for pairwise comparisons.

\subsubsection{Speed}

We measured text entry speed in Chinese characters per minute (CPM). Participants used Pinyin \cite{Website-Pinyin}, a phonetic spelling system in Roman characters to input Chinese characters. To input a Chinese character, users type the Pinyin of the desired character (2 - 6 letters) and then select the target from a candidate list. Participants can also type the Pinyin of a Chinese word, consisting of two to four Chinese characters, and then select the whole word. In short, the process of inputting a Chinese character/word is similar to inputting an English word with word prediction/correction. We measured typing speed in CPM with this formula:

\begin{equation}
	CPM = \frac{|S|}{T} \times 60
\end{equation}

where |S| is the length of the transcribed paragraph in characters (including punctuation), and T is the completion time, i.e., the elapsed time in seconds from the first intentional touch to the last one. All the time consumption, including the time of candidate selection, was taken into account.

%我们使用的是中文输入法,我们所统计的量是中文字/分钟,公式是,当用户输入数字或字母时,其速度不纳入统计范围。我们测试了五种不同的输入任务,值得注意的是,只有transcription任务中的输入速度比较符合用户的ceiling speed,而在其它任务中,完成任务的时间包含了用户思考的时间和键鼠切换的时间。

\begin{figure}[!tbh]
	\includegraphics[width=0.6\linewidth]{figures/speed.png}
	\centering
	\caption{Text entry speed of the three keyboards over sessions. Error bars indicate 95\% confidence interval.}
	\label{fig:speed}
\end{figure}

Figure \ref{fig:speed} shows the users' typing speeds over sessions. There is no significant effect of session on speed ($F_{4,56}=1.76,p=.15,\eta_p^2=0.11$). The result indicates that the learning costs of the three keyboards were low. Participants reached the ceiling performance after a five-minute training. Keyboard has a significant effect on speed ($F_{2,28}=26.76,p<.001,\eta_p^2=0.66$). Pair-wise comparisons show significant differences between all the keyboard pairs: ordinary keyboard vs. TypeBoard ($p<.005$), ordinary keyboard vs. TypeBoard plus ($p<.001$), and TypeBoard vs. TypeBoard plus ($p<.005$). The participants' average typing speed on the ordinary keyboard was 43.71 CPM (SD = 6.52). The typing speed on the TypeBoard was 48.87 CPM (SD = 10.14), outperforming the ordinary keyboard by 11.78\%. The typing speed on the TypeBoard plus was 52.97 CPM (SD = 9.85), outperforming the ordinary keyboard by 21.19\%. Results show that the TypeBoard improves the efficiency of the touchscreen keyboard.

\subsubsection{Error rate}

We used two metrics to measure text entry accuracy: (1) Uncorrected Error Rate (UER) - text entry errors that remain in the transcribed string. UER is the number of uncorrected erroneous Chinese characters divided by the number of correct and erroneous characters. (2) Corrected Error Rate (CER) - text entry errors that are fixed (e.g., backspaced) during entry. CER is the number of corrected erroneous Chinese characters divided by the number of correct and erroneous characters. The corrections of Pinyin while inputting a word were not taken into account of CER. As UER and UER violated the normalcy, we used the Aligned Rank Transform for nonparametric factorial analysis \cite{2011-Aligned}.

\begin{figure}[!tbh]
	\includegraphics[width=1.0\linewidth]{figures/error_rate.png}
	\centering
	\caption{Uncorrected error rates and Corrected error rates of the three keyboards over sessions.}
	\label{fig:error_rate}
\end{figure}

Figure \ref{fig:error_rate} shows the CER and the UER over sessions. There is no significant effect of session on CER ($F_{4,56}=1.01,p=.39$). Keyboard has a significant effect on CER ($F_{2,28}=9.49,p<.005$). Pair-wise comparisons showed significant differences between the following keyboard pairs: ordinary keyboard vs. TypeBoard ($p<.01$), and ordinary keyboard vs. TypeBoard plus ($p<.005$). The average CERs of the ordinary keyboard, TypeBoard, and TypeBoard plus were 6.66\% (SD = 4.42\%), 4.58\% (SD = 3.58\%), and 4.21\% (SD = 2.58\%).
There is no significant effect of session ($F_{4,56}=0.41,p=.71$) or keyboard ($F_{2,28}=0.001,p=.998$) on UER. The average UERs of the ordinary keyboard, TypeBoard, and TypeBoard plus were 1.29\% (SD = 1.67\%), 1.28\% (SD = 1.38\%), and 1.28\% (SD = 1.16\%).
Results show that the TypeBoard reduces the probability of making a typo compared with the ordinary keyboard. This is the main reason the TypeBoard improves the typing speed of the touchscreen keyboard. The TypeBoard plus does not reduce the probability of making a typo compared with the TypeBoard. Thus, there are other reasons for faster typing on the TypeBoard plus.

%在中文输入法中,CER可能会比其它语言的输入法更多,这是因为用户可能会利用词语的联想功能来打一个汉字,然后把后面那个字删掉。

%Figure 14 shows the UER and the CER over five days. There is no significant effect of model or day on UER. The average UER was 1.17% (SD = 1.02%) for the general model and 1.50% (SD = 1.40%) for the personal model. There is no significant effect of the model on CER. The days have a significant effect on CER (F4,56 = 6.84,p < .005). Pair-wise comparisons showed significant differences between the following day pairs: 1-3(p<.005), 1-4(p<.05), 1-5(p<.005), 2-3(p<.05) and 2-5(p<.05). On day 5, the average CER was 3.22% (SD = 2.92%) for the general model and 2.92% (SD = 1.65%) for the personal model. That is, participants made corrections once every 30 words with both of the models.

\subsubsection{Time components}

Time components.

\begin{figure}[!tbh]
	\includegraphics[width=0.6\linewidth]{figures/time_components.png}
	\centering
	\caption{Time components.}
	\label{fig:time_components}
\end{figure}

\subsubsection{Unintentional touches}

For the three configurations ordinary keyboard, TypeBoard and TypeBoard plus, the proportions of unintentional touches were xx.xx\%, xx.xx\% and xx.xx\% respectively.

English.

为了方便计算,我们在上述统计中假设算法100\%正确预测了点击的有意性。

统计发现,显著性影响。

讨论造成显著性的原因。

在实验三中,TypeBoard上的误触与实验二相比显著降低,我们认为,这是任务不同造成的。在誊写这种快速输入的任务下,用户在普通触屏键盘和TypeBoard上都没有必要将手指休息在键盘上。但在TypeBoard+tactile设置下,用户在誊写任务下仍然引发了大量的“误触”,这暗示,用户可能并不只是将手指休息在触屏上,而是尝试利用触屏上的纹理来对齐他们的手指,从而达到部分的盲打。

\subsubsection{Touch position}

%Figure xx illustrates the multiple finger resting behavior through point clouds. The distribution seems optional on the ordinary TypeBoard, and seems regular on the TypeBoard plus tactile landmarks. xx.x\% of the resting touchpoints laid on the second row of keys on the TypeBoard with landmarks, which has an significant difference (F, p) from the ordinary TypeBoard (xx.x\%). This indicates that users leveraged the tactile landmarks to align their fingers.
%【图:用户在multiple finger resting时的手指点云,TypeBoard with/without landmarks】

Figure xx shows the distribution of intentional touchpoints over keyboard configurations. We used xxx to cluster the point cloud of each key. xxx shows that the point cloud obeyed the 2D Gaussian distribution (F, p). The average standard deviation of the distributions were xx.x mm (SD=), xx.x mm (SD=) and xx.x mm (SD=). RM-ANOVA shows that the keyboard has a significant effect on users' touching accuracy (F, p). Users typed more accurately on the TypeBoard plus landmarks compared with the regular keyboard (p) and the TypeBoard (p). The analysis of touch position shows that users aligned their fingers on the TypeBoard with tactile landmarks, which improved their typing accuracy.

【图:各实验设置下用户有意点击点云】

\subsubsection{Subjective Rating and Feedback}

English.

物理负担(疲劳程度),心理负担,主观输入速度,主观输入准确率(误触准确率,而非选词准确率)。

\subsection{Discussion}

\subsubsection{TypeBoard vs. Regular Keyboard}

Compared with regular touchscreen keyboards, the TypeBoard has the advantages of avoiding fatigue and improving subjective user experience.

(1)避免疲劳。TypeBoard用户在誊写任务的每100次打字行为中,系统就会阻止xx.x次多指休息行为和xx.x次小鱼际点击。实验二表面,TypeBoard在其它需要更多键鼠切换或者思考的文本输入任务下,所阻止的多指休息次数会更多(xx.x 每100次点击)。结果表明,用户在TypeBoard上会主动地利用可以将手指和手腕休息在touchscreen上的特性。而且在用户的主观反馈当中,用户也表示TypeBoard显著地降低了他们的疲劳程度。
(2)主观用户体验。TypeBoard在主观输入速度、主观输入准确率等方面都显著优于普通键盘。这说明TypeBoard可以提高用户体验。

我们推断,TypeBoard的优势主要在降低错误率,和避免疲劳。

\subsubsection{TypeBoard vs. TypeBoard plus}

English

TypeBoard plus 是我们设想中的一种软键盘形态,即利用可变形触屏[xx]或者添加layout[xx]的方法来达到提供触觉landmarks的目的。我们发现,TypeBoard plus不仅能够降低疲劳、提高用户体验,还能够显著地提高软键盘上的文本输入速度,将输入速度提高xx.x\%。输入效率提高的主要原因是用户可以在TypeBoard plus中align fingers,从而实现盲打。

我们推断,TypeBoard的优势主要在盲打。


\section{Limitation and Future Work}


\section{Conclusion}

一些结论。

\bibliographystyle{ACM-Reference-Format}
\bibliography{acmart}

\end{CJK*}
\end{document}
\endinput
%%
%% End of file `sample-acmlarge.tex'.
