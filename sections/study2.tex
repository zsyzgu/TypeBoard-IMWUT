\section{Study 2: User Behavior on TypeBoard}

实验二的目的是研究用户在防误触键盘上的打字行为,并根据分析此行为优化触屏键盘上的防误触算法。在实验一之后,我们已经得到了初版的TypeBoard算法,在本实验中,我们采集了用户在TypeBoard防误触触屏上的打字数据。实验共采集了5组*4人=20人的数据,在采集完每4名被试之后,我们都会利用新的数据去优化TypeBoard,并在之后的实验中使用优化后的TypeBoard,这样做的目的是迭代地逼近鲁棒的防误触算法及其上的用户行为模型。

\subsection{Participants}

我们从本地的校园中邀请了20名用户参与实验,他们的年龄从xx岁到xx岁不等,平均数是xx,标准差是xx,其中有xx名女性。所有的用户都是右撇子,所有用户有超过xx年的手机文本输入经历,xx名用户常用平板电脑进行文本输入。这些用户没有参与过实验一。如上面所说,我们将用户分成了五组,为了保证五组用户的平衡性,我们通过程序随机模拟了10000次分组,最终确定了平均年龄最相近的分组方式,这五组的年龄平均数分别为xx,xx,xx,xx,xx,每组分别有xx,xx,xx,xx,xx名女性。

\subsection{Design and Procedure}

在实验之前有一个训练阶段,用户有五分钟时间通过誊写例句熟悉该键盘,例句随机选自xx例句库[xx]。由于所有用户都没有过使用防误触触屏键盘的体验,在用户的试用过程中,实验者提醒用户该键盘有防误触的功能,平时可以把手休息在触屏上。这不是强制的要求,具体行为由用户根据具体的任务和自己的喜好决定。

正式实验分为四个session,分别收集了用户在填写个人信息、描述个人爱好、模拟开卷考试和看图写话这四个任务下的打字数据,我们通过拉丁方来平衡每个用户做这四个任务的顺序。如图xx所示展示了实验二的实验设置,桌面上的实验设备包含morph-sensel压力触摸板、鼠标、显示器和耳机。在本实验中,用户同样通过填写word文档的方法来完成四个文本输入任务(图xx)。和实验一不一样的是,用户在本实验中真的能够通过TypeBoard来键入字母,且打字的过程中能够听到“啪啪”的声音反馈。在极少数情况下,用户会发现一类机器学习总是错判的情况,字母不能正常上屏。举例来说,第x名用户经常在单手五指都休息的情况下,另一只手打字,系统总是漏报。在这种情况下我们鼓励用户维持这种系统会误判的用户行为,并像实验一中一样想象字母已经正确上屏,然后在标注的时候标注出这些错误。

在每个session结束后,用户通过交互式程序标注刚刚完成的session的报点。和实验一的标注过程相同,所有的报点一开始都被标为红色(误触),用户需要将每一个有意的点击标注成绿色(正例)。实验者在一旁观察标注的过程,并通过另一台电脑事先得知机器学习的结果,当发现机器学习结果和用户人工标注不符的情况时,实验者会人工分析其中的原因,如果实验者不能通过独立思考弄明白其中的原因,或者他认为被试标注出错时,他会立即和被试进行讨论。

在本实验中,用户完成输入任务的总时间大约为30分钟,比实验一稍长,这是因为实验二的输入过程涉及键入字母和删改操作;用户完成标注的时间大约为45分钟;每两个session之间用户休息5分钟时间以避免疲劳,实验总时长为90分钟。

由于用户的打字行为和触屏防误触能力会互相影响,我们将20名被试平均分为五组,在每组被试完成实验后,我们都会更新数据集和添加必要的特征向量,以训练新的防误触算法,迭代地增强触屏键盘的防误触能力。这就是说,我们期望每组用户所用到的TypeBoard键盘的防误触能力都会更强,也期望每组用户的打字行为更接近防误触键盘上的自然输入行为。

\subsection{Apparatus}

实验二的设备与实验一相比,只多出了一个普通的有线耳机,用于提供打字时的声音反馈。实验二所运行的程序包含两个功能,一是区分打字事件和误触,二是将打字事件触点位置上的字母上屏,其运行速度为50FPS。

\subsection{Result}

在实验的过程中,每五个用户的实验结束之后,我们都会采用新的数据来优化防误触算法,如图xx所示是防误触算法随着完成实验人数的增加的变化,其中每个测试点的数据集包含实验二过程中已经收集到的数据加上实验一的所有数据,评测方法是leave-one-out检测。结果表明,随着实验人数的增加,迭代更新TypeBoard、初版TypeBoard和baseline[xx]的差距在显著拉大,这说明TypeBoard的防误触能力在增强,我们采集到的数据也越来越接近用户在一个完美防误触键盘上的打字行为。

【图:baseline、初版TypeBoard、迭代更新TypeBoard随着实验人数增加,准确率的变化】

在20名用户的实验都完成了之后,抛去用户无法区分的xx个数据点,我们共收集了xx个有效的数据点。在有效的数据中,正例(打字)的比例是xx.x\%,负例(误触)的比例是xx.x\%。我们总结了一些实验一的机器学习算法不能很好处理的fail-cases(如表xx左侧所示),我们针对这些情况专门设计了相应的特征组(如表xx右侧所示)。最终,我们的算法在实验二的数据集上达到了xx.x\%的准确率,而baseline的准确率仅为xx.x\%,TypeBoard和baseline的差距与实验一相比显著增大。

【表:实验二新总结出来的fail-cases和应对方案】

这一结果说明:第一,用户在防误触键盘上打字时,存在更多、更难分辨的误触点,对防误触算法带来更多的挑战;第二,在该用户行为下,结合压力信息的机器学习方法能够鲁棒地防误触,而baseline-基于电容屏信息和阈值地方法变得不可用。

用户的主观评分和采访。评分包括主观的好评率over 任务,主观的休息手指频率over 任务。

\subsection{Discussion}

误触点的构成:休息、手掌误触、输入时误触的比例。

实验任务显著影响了用户行为。这说明了考虑不同实验任务的必要性。

TypeBoard算法在不同延迟下的识别准确率问题。

TypeBoard算法在不同数据组合下的准确率问题。motivation,比如有的设备没有压力信息,能否正确区分误触与否?

技术实现和用户行为之间的互相影响,及应对方案。可以列举很多相关工作进行讨论,比如其它防误触算法、文本输入相关算法。
