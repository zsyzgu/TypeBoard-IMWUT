\section{Study 2: User Behavior on TypeBoard}

在这个实验中,我们采集了用户在一个有反馈的防误触触屏键盘上的打字数据,系统在识别到用户打字时会给出声音反馈,并让字母上屏,系统识别到误触时不会作出反馈。我们在压力触摸板上贴上了Qwerty布局的贴纸,系统根据该布局选字母上屏,因此用户可以准确输入每个字母。本实验的目标是研究用户在防误触键盘上的打字行为,从而指导触屏键盘防误触算法的改进。与实验一一样,用户五个不同的文本输入任务下打字。由于用户的打字行为和触屏的防误触能力会互相影响,我们在每名被试完成实验之后,都会更新数据集,重新训练防误触算法,让下一名用户能使用防误触能力更强的触屏键盘。实验共采集了xx个数据点,其中xx.x\%是误触点,在该数据集上,我们改进的防误触算法准确率达到了xx.x\%,相比之下,baseline[xx]的准确率仅为xx.x\%,差距与实验一相比进一步拉大。这一结果说明两点:首先,用户在一个真的能防误触的键盘上打字时,会比空想引发更多的难以分辨是否有意的触摸点,对防误触算法带来更多的挑战;第二,结合压力触摸屏上众多的传感信息的机器学习方法仍然能胜任该严格的数据集,而baseline则变得不可用。

\subsection{Design and Procedure}

我们从本地的校园中邀请了16名用户参与实验,他们的年龄从xx岁到xx岁不等,平均数是xx,标准差是xx,其中有xx名女性。所有的用户都是右撇子,所有用户有超过xx年的手机文本输入经历,xx名用户常用平板电脑进行文本输入。这些用户没有参与过实验一。在实验之前有一个训练阶段,用户有五分钟时间输入例句熟悉该键盘。由于所有用户都没有过使用防误触触屏键盘的体验,我们提醒用户该键盘有防误触的功能,平时可以把手休息在触屏上,但这不是强制的要求,具体行为由用户根据具体的任务和自己的喜好决定。如图xx所示展示了实验二的实验设置,与实验一相同,桌面上的实验设备包含morph-sensel压力触摸板和一个显示器。与实验一不同的是,我们在压力触摸板上加上了Qwerty布局的贴纸,用户提示用户每个字母的位置,输入法实际上也会根据该布局将用户的点击识别成相应的字母并上屏;另外,用户戴上了耳机来接收声音反馈,当系统识别到打字事件时,会发出类似iPhone上打字的啪啪声,当系统识别到误触点击时,不会给出任何反馈。

实验二在实验任务的设置上与实验一相同,共有五个session,分别是填写个人信息、描述个人爱好、模拟开卷考试、看图写话和誊写这五个任务。在每个session结束后,用户需要标注刚刚完成的session的报点。对于用户在输入过程中没有出现删改的时间段,我们认为机器学习正确地预测了报点是否为有意输入,这部分数据我们直接采用了机器学习的结果作为标注;而对于用户输入过程中出现了删改的时间段,我们认为有可能出现了机器学习预测出错的情况,对于这些数据点(敏感数据),我们调用了与实验一类似的程序,让用户去人工标注——将每个报点标注为正例、负例或者剔除出数据集。在用户标注敏感数据的过程中,实验者会在一旁观察,通过独立思考或者和被试讨论的方式来尝试弄清楚系统每个错判的原因。在本实验中,用户完成输入任务的总时间大约为30分钟,比实验一稍长,这可能是因为实验二的输入过程设计删改操作;用户完成标注的时间大约为30分钟;每两个session之间强制休息5分钟时间以避免疲劳,实验总时长为80分钟。

\subsection{Apparatus}

实验二的设备与实验一相比,只多出了一个普通的有线耳机。实验二所运行的程序包含两个功能,一是区分打字事件和误触,二是将打字事件触点位置上的字母上屏,其运行速度为50FPS。

\subsection{Result}

在实验的过程中,每个用户的实验结束之后,我们都会采用新的数据来优化防误触算法,如图xx所示是防误触算法随着完成实验人数的增加的进步,这一迭代增强的模型的分类准确率显著高于baseline算法[xx]和初版TypeBoard算法,训练集和测试集应该怎么算?。图中的结果说明,随着实验人数的增肌,TypeBoard与baseline的差距在显著增加,这说明TypeBoard的防误触能力在增强。如此一来,我们采集到的数据也是越来越接近用户在一个能够很好防误触的键盘上打字的行为。

【图:baseline、初版TypeBoard算法、迭代更新算法over-participants】

在十六名用户的实验都完成了之后,数据共包含xx个数据点,其中用户无法区分的点的占比为xx.x\%,刨去这些数据点之后正例(打字事件)的比例是xx.x\%,负例(误触点)的比例是xx.x\%。我们总结了一些实验一的机器学习算法不能很好处理的fail-cases(如表xx左侧所示),我们针对这些情况专门设计了相应的特征组(如表xx右侧所示)。最终,我们的算法在实验二的数据集上达到了xx.x\%的准确率,而baseline的准确率仅为xx.x\%,我们与baseline之间的差距较实验一相比继续拉大了,我们将此时的防误触算法作为终版的TypeBoard算法。至此,我们已经得到了令人满意的防误触算法。

【表:实验二新总结出来的fail-cases和应对方案】

用户的主观评分和采访。

\subsection{Discussion}

误触点的构成:休息、手掌误触、输入时误触的比例。

实验任务显著影响了用户行为。这说明了考虑不同实验任务的必要性。

TypeBoard算法在不同延迟下的识别准确率问题。

TypeBoard算法在不同数据组合下的准确率问题。motivation,比如有的设备没有压力信息,能否正确区分误触与否?
