\section{Study 2: User Behavior on TypeBoard}

在这个实验中,我们采集了用户在有反馈的防误触触屏键盘上的打字数据,系统在识别到用户打字时会让字母上屏,并给出声音反馈。系统不会对误触作出反馈。本实验的目标是研究用户在防误触键盘上的打字行为,从而指导触屏键盘防误触算法的改进。与实验一一样,用户五个不同的文本输入任务下打字。由于用户的打字行为和触屏防误触能力会互相影响,我们将25名被试平均分为五组,在每组被试完成实验后,我们都会更新数据集以训练新的防误触算法,迭代地增强触屏键盘的防误触能力。实验共采集了xx个数据点,其中xx.x\%是误触点,在该数据集上,改进的防误触算法准确率达到了xx.x\%,相比之下,baseline[xx]的准确率仅为xx.x\%,差距进一步拉大。结果说明:首先,用户在防误触键盘上打字时,存在更多、更难分辨的误触点,对防误触算法带来更多的挑战;第二,在该用户行为下,结合压力信息的机器学习方法能够鲁棒地防误触,而baseline-基于电容屏信息和阈值地方法变得不可用。

\subsection{Design and Procedure}

我们从本地的校园中邀请了25名用户参与实验,他们的年龄从xx岁到xx岁不等,平均数是xx,标准差是xx,其中有xx名女性。所有的用户都是右撇子,所有用户有超过xx年的手机文本输入经历,xx名用户常用平板电脑进行文本输入。这些用户没有参与过实验一。在实验之前有一个训练阶段,用户有五分钟时间输入例句熟悉该键盘。由于所有用户都没有过使用防误触触屏键盘的体验,我们提醒用户该键盘有防误触的功能,平时可以把手休息在触屏上,但这不是强制的要求,具体行为由用户根据具体的任务和自己的喜好决定。如图xx所示展示了实验二的实验设置,与实验一相同,桌面上的实验设备包含morph-sensel压力触摸板和一个显示器。与实验一不同的是,输入法会将用户的点击识别成相应的字母上屏。另外,用户戴上耳机接收声音反馈-类似iPhone上打字的啪啪声。

实验二在实验任务的设置上与实验一相同,共有五个session,分别是填写个人信息、描述个人爱好、模拟开卷考试、看图写话和誊写这五个任务,我们通过拉丁方来平衡每个用户做这五个任务的顺序。实验过程中,用户用我们的键盘完成五个任务的word文档。在用户发现触屏键盘发生误触的时候,他需要告诉实验者,实验者会把时间戳记录下来。在极少数情况下,用户会发现一类机器学习总是错判的情况,字母不能正常上屏。举例来说,第x名用户经常在单手五指都休息的情况下,另一只手打字,系统总是漏报。在这种情况下我们鼓励用户维持这种系统会误判的用户行为,并像实验一中一样想象字母已经正确上屏,然后在标注的时候纠正这些错误。

在每个session结束后,用户通过一个交互式程序标注刚刚完成的session的报点。与实验一相同,该交互式程序同时展示了实验过程中的录屏视频和压力板图像,用户通过鼠标左键、右键和中键将压力板图像中的报点标记为正例、负例和剔除;与实验一不同的是,压力板图像中的报点一开始被标注成了机器学习预测的结果。为了节省用户的时间和专注力,标注程序只要求用户标记可疑的时间段的数据,可疑时间段包括两类:一是用户在实验过程中自行发现了误判,并且告诉了实验者;二是用户删改了文本的时间段。在用户标注敏感数据的过程中,实验者会在一旁观察,通过独立思考或者和被试讨论的方式来尝试弄清楚系统每个错判的原因。在本实验中,用户完成输入任务的总时间大约为30分钟,比实验一稍长,这是因为实验二的输入过程涉及删改操作;用户完成标注的时间大约为30分钟;每两个session之间用户休息5分钟时间以避免疲劳,实验总时长为80分钟。

\subsection{Apparatus}

实验二的设备与实验一相比,只多出了一个普通的有线耳机。实验二所运行的程序包含两个功能,一是区分打字事件和误触,二是将打字事件触点位置上的字母上屏,其运行速度为50FPS。

\subsection{Result}

在实验的过程中,每五个用户的实验结束之后,我们都会采用新的数据来优化防误触算法,如图xx所示是防误触算法随着完成实验人数的增加的进步,其中每个测试点的数据集包含实验二过程中已经收集到的数据加上实验一的所有数据,评测方法是leave-one-out检测。结果表明,随着实验人数的增加,迭代更新TypeBoard、初版TypeBoard和baseline[xx]的差距在显著增加,这说明TypeBoard的防误触能力在增强,我们采集到的数据也越来越接近用户在一个完美防误触键盘上的打字行为。

【图:baseline、初版TypeBoard、迭代更新TypeBoard随着实验人数增加,准确率的变化】

在十六名用户的实验都完成了之后,数据共包含xx个数据点,其中用户无法区分的点的占比为xx.x\%,刨去这些数据点之后正例(打字事件)的比例是xx.x\%,负例(误触点)的比例是xx.x\%。我们总结了一些实验一的机器学习算法不能很好处理的fail-cases(如表xx左侧所示),我们针对这些情况专门设计了相应的特征组(如表xx右侧所示)。最终,我们的算法在实验二的数据集上达到了xx.x\%的准确率,而baseline的准确率仅为xx.x\%,我们将此时的防误触算法作为终版的TypeBoard算法。

【表:实验二新总结出来的fail-cases和应对方案】

用户的主观评分和采访。

\subsection{Discussion}

误触点的构成:休息、手掌误触、输入时误触的比例。

实验任务显著影响了用户行为。这说明了考虑不同实验任务的必要性。

TypeBoard算法在不同延迟下的识别准确率问题。

TypeBoard算法在不同数据组合下的准确率问题。motivation,比如有的设备没有压力信息,能否正确区分误触与否?

技术实现和用户行为之间的互相影响,及应对方案。可以列举很多相关工作进行讨论,比如其它防误触算法、文本输入相关算法。
