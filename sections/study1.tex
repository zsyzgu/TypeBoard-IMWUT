\section{Study 1: User Behavior on Imaginary TypeBoard}

在这个实验中,我们采集了用户在一个没有反馈的触屏键盘上的打字数据,实验中我们要求用户想象该键盘可以防误触。本实验的目标是研究用户在想象中可以防误触的键盘上打字的行为,从而指导触屏键盘防误触算法的设计。用户在填写个人信息、描述个人爱好、模拟开卷考试、看图写话和誊写这五个不同的文本输入任务下打字。由于(不正确的)反馈可能会严重影响用户的输入行为,在实验一的阶段我们要求用户在无反馈的键盘上打字,用户不能真的看到字母上屏,他们需要想象字母能够上屏。实验共采集了xx个数据点,其中xx.x\%是误触点,在该数据集上,我们开发了基于机器学习的防误触算法,Leave-one-out准确率达到了xx.x\%,相比之下,baseline[xx]的准确率仅为xx.x\%。这一结果说明两点:首先,用户在想象中能防误触的键盘上打字时,会引发很多难以仅通过阈值方法区分有意与否的触摸点,对防误触算法带来挑战;第二,结合压力触摸屏上众多的传感信息、结合时间、空间信息,能够大幅度提高触屏上误触点的识别。

\subsection{Design and Procedure}

我们从校园中邀请了15名用户被试,年龄从xx到xx不等,平均数xx,标准差xx,xx名女性。所有的被试都是右撇子,所有用户有超过xx年的手机文本输入经历,xx名用户常用平板电脑进行文本输入。图xx展示了实验的设置,桌子上放置着一块morph-sensel压力触摸板和显示器,用户可以根据自己的需求调整椅子的高度、压力触摸板和显示器的位置。压力触摸板上贴了一张展示26键Qwerty键盘布局的贴纸,用于提示用户每个按键的位置。

标注:Annotation

实验分为5个session,分别收集了用户在填写个人信息、描述个人爱好、模拟开卷考试、看图写话和誊写这五种不同的文本输入任务下的打字数据,我们通过拉丁方来平衡每个用户做这五个任务的顺序。在输入任务中,由于触屏没有任何反馈,用户不能真的看到字母上屏,而是想象字母上屏了。用户是通过填写一个word文档来完成这五个文本输入任务的,图xx展示了这个word任务文档中每种任务的例子,每个任务的具体描述如下。

\begin{enumerate}
	\item{\textbf{填写个人信息任务:}word文档中有一个表格,其中包含用户姓名、性别、专业等问题。用户操作触摸板(想象中的TypeBoard键盘)和鼠标填写表格。为了保护用户的隐私,用户可以填写错误的信息,前提是他能记住他填写的内容,以便标注时作为参考。用户需要通过鼠标将输入指针移动到表格相应的位置下,然后再触摸板上敲击“输入”,想象所需字母被填写在了word文档中。该任务模拟了真实场景中,需要频繁切换键鼠操作的输入类型。}
	\item{\textbf{描述个人爱好任务:}word文档中有一个问卷,其中包含“最喜欢的城市”、“最喜爱的食物”等个人喜好相关的问题。和任务一相同,用户通过键鼠配合操作完成问卷的填写。该任务模拟了真实场景中简单的问卷填写任务[xx]。}
	\item{\textbf{模拟开卷考试任务:}word文档中包含若干有一定难度的知识性问题,如“比利时的首都在哪里”、“元素周期表中第50号元素时?”。在此任务中,用户很可能不知道问题的正确答案,此时她需要通过搜索引擎来得到答案。用户在使用搜索引擎时存在一个问题是,她输入的文字并不能真的上屏,此时我们要求用户边输入边把搜索的关键词说出来,实验者帮忙使用键盘完成搜索。both搜索的过程和填写试卷的过程是本实验采集的数据。该任务模拟了真实场景中常见的搜索引擎任务。}
	\item{\textbf{看图写话任务:}word文档中包含了一张简笔画(如图xx的xx所示),用户需要根据这幅图写一个五句话的小作文。在此任务中,我们同样要求用户边说边写,实验者在一旁进行速记,这是为了在标注中给用户提高上下文。该任务模拟了真实场景中常见的创作型文本输入[xx],这一类任务有着xx的特点。}
	\item{\textbf{誊写任务:}word文档中包含了一句话,用户需要尽可能快地将这句话誊写五次。每个用户所誊写的句子都是不同的,都是从phrase-set[xx]中随机抽取。誊写任务是文本输入工作中最常见的评测任务,适用于评测输入法的输入效率上限。}
\end{enumerate}

在每个session结束后,用户需要通过一个交互式程序来标注刚刚所进行session的报点数据,该程序同时展示了实验过程中的录屏视频和压力板图像,用户需要结合压力板图像和录屏视频提供的上下文信息,给压力板图像中的每个报点标注。如图xx所示,压力图像中所包含的报点初始默认值为负例(误触),用红色点表示。用户若认为一个报点是正例(打字事件),则需要用鼠标左键将该点标为绿色;用户可以通过鼠标右键将报点重新标为红色;用户若不清楚该报点是正例还是负例,则需要用鼠标中键将该点标为蓝色,表示剔除出数据集。本实验的输入任务部分,五个session加起来大约需要15分钟,标注过程大约45分钟,每两个session之间强制休息5分钟时间来避免疲劳,实验总时长为80分钟。

\subsection{Apparatus}

Morph Sensel简介,包含多少像素的压力数据,压力数据的范围是xx,精度是xx。自带报点功能,多大的触摸点、多重的点击算是一个报点,压力的报点阈值比较低,不会出现漏报正例的情况。

Morph Sensel的长和宽分别是xx厘米,触摸板上展示Qwerty布局的贴纸的长度与压力板相贴合(也是xx厘米),高度为xx厘米。这一Qwerty布局的26键位置比15寸的MacBook的Qwerty键盘略小,为xx.x\%的等比例缩放。由于软键盘的布局是多种多样的,在这份工作中,我们不会将键盘布局相关的先验知识作为防误触算法的特征值。

电脑设置,运行速度-->程序运行帧率是稳定的50FPS。

\subsection{Result}

原始数据共包含xx个数据点,其中用户无法区分的点的占比为xx.x\%,刨去这些数据点之后正例(打字事件)的比例是xx.x\%,负例(误触点)的比例是xx.x\%。我们先使用简单的机器学习方法对数据进行验证,当机器学习结果和用户标注结果有大量不同时,我们会将用户召回,让她重新标注这些分类错误的数据,其中,有xx.x\%的数据的确是用户标错了(有的用户对误触的理解出现了偏差),剩下xx.x\%的数据是机器学习预测错误,其中xx.x\%的误触和xx.x\%的漏报。我们对这些简单的机器学习不能正确分类的数据点进行了人工观察,总结出来这些无触点的分类如下表所示。

【表格:简单机器学习容易分类出错的误触点类型】

从表格xx能够看出来,大多数误触点都是有明显的特点的,我们针对几乎每种错判,都设计了专门的特征向量,用于机器学习。只有表中加粗的两类错判很难找到特征,这一类数据的占比为xx.x\%,用户仅能通过任务的上下文来标注,而系统不可能准确知道用户想要输入的文字,因此我们认为这一部分数据的分类问题是不可解的。因此在该数据集上,机器学习的准确率上限(贝叶斯概率)大概在xx.x\%,这也是我们recognition的目标。

我们采用支持向量机来实现二分类模型,特征向量如表格xx中间的一列所示,每一段特征向量独立的预测准确率也公布在表格中了。将所有特征结合起来的准确率为xx.x\%。进一步的,由于空间相邻触点特征组、时间相邻触点特征组和触点位置和位移特征组之间比较独立,我们用触摸点自身特征组分别结合上述三组特征,训练了三个二分类模型,再用投票算法[xx]将三个模型融合起来,准确率进一步提高至xx.x\%,接近贝叶斯概率。

\subsection{Discussion}

误触点的构成:休息、手掌误触、输入时误触的比例。

实验任务显著影响了用户行为。

初版TypeBoard算法的实际使用体验(有反馈)。存在问题:数据集还不够全面,在有反馈的使用场景下,存在一些可复现的误触。因此,我们有必要设计实验二,来调查用户在有反馈的触屏键盘上的打字规律。
