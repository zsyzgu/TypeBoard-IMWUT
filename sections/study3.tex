\section{Study 3: Evaluation on TypeBoard}

The motivations of study three were two-fold. First, we evaluated the performance and user experience of TypeBoard on different text entry tasks. The baseline was the software keyboard without unintentional touch rejection. Second, as we introduced in related work, tactile landmarks on keyboards improve users' typing speed by enabling touch type. In this study, we investigated the feasibility of TypeBoard plus tactile landmarks. In summary, we evaluated users' typing experience on three settings: (1) regular software keyboard, (2) TypeBoard, and (3) TypeBoard with tactile landmarks.

%实验三的目的有两点:第一,评测TypeBoard的性能,包括在不同的文本输入任务下的输入效率和主观用户体验,baseline是没有防误触算法的触屏键盘。第二,相关工作中我们提到,在防误触触屏键盘上,有望通过加上纹理反馈帮助用户盲打,提高输入效率,本实验探索了这一假设是否成立。

\subsection{Participants}

We recruited 15 participants from the campus (aged from xx to xx, M = xx, SD = xx, xx females). All the participants were right handed. The have used software keyboards on smartphones for not less than xx years. xx of the 15 participants were familiar with tablet keyboards. Participants did not take part in the last two experiments.

%我们从本地的校园中邀请了15名用户参与实验,他们的年龄从xx岁到xx岁不等,平均数是xx,标准差是xx,其中有xx名女性。所有的用户都是右撇子,所有用户有超过xx年的手机文本输入经历,xx名用户常用平板电脑进行文本输入。这些用户没有参与过前两个实验。

\subsection{Design and Procedure}

The study followed a within-subject design to compare the typing experience in three keyboard setting (figure xx).

\begin{enumerate}
	\item{\textbf{Config. 1):} \emph{Regular Software Keyboard.} This setting represent the tablet keyboard we use in daily life. All contacts on the key are considered as keystrokes. Users need to hang their wrists in the air to avoid unintentional touches.}
	\item{\textbf{Config. 2):} \emph{TypeBoard.} TypeBoard is a software keyboard with unintentional touches rejection. Only intended touches are considered as keystrokes. Users can rest their hand on the keyboard.}
	\item{\textbf{Config. 3):} \emph{TypeBoard + Tactile Landmarks.} To afford tactile landmarks on TypeBoard, we attached 0.05mm thick stickers to the position of each key. There were small bumps on the F and J keys, which is the same as the physical keyboard. Users could position their hands without having to look at the keyboard.}
\end{enumerate}

%我们采取了within-subject的实验方法,对比了三种不同的实验设置(如图xx所示):
%\begin{enumerate}
%	\item{\textbf{Config. 1):} \emph{Regular Soft Keyboard.} 没有防误触功能的软键盘。}
%	\item{\textbf{Config. 2):} \emph{TypeBoard.} TypeBoard,有防误触功能的软键盘,这也是实验二的设置。}
%	\item{\textbf{Config. 3):} \emph{TypeBoard + Tactile Landmarks.} 在TypeBoard的基础上,在每个按键的位置上贴上了0.05毫米厚的贴纸,其中F键和J键上各再加上了一条0.05毫秒厚的横杠,用于模拟物理键盘上的触觉Lankmakrs。0.05毫米的厚度可以让用户明显地摸到每个键的边缘,同时符合可变形触摸屏的工艺要求[xx]。}
%\end{enumerate}

The experiment contained three sessions of keyboard settings. In each session, participants filled in a document to complete five text entry tasks, including the four tasks we had introduced in the last two studies. In this study, we had an additional transcription task, where participants transcribed a Chinese phrase for five times as quick and accurate as possible. The transcription task is widely used in text entry studies [xx,xx,xx] to evaluate the ceiling typing speed. We counterbalanced both the order of keyboard settings and the order of tasks using a balanced latin square.

%每名用户使用这三种实验设置来完成五个文本输入任务,除了前两个实验中所述的四个任务(填写个人信息、描述个人爱好、模拟开卷考试、看图写话)以外,实验三中新增了文本誊写任务,具体是誊写一个中文句子5次(某中文打字测速网站的第一句话)。文本誊写任务在文本输入相关工作中很常见,一般用于测试文本输入法的输入效率上限[xx,xx,xx,xx]。为了平衡学习效应,我们采用拉丁方来规定用户使用这三种实验设置的顺序,和完成五个输入任务的顺序。

【图:实验三要对比的三种实验设置】

Participants had five minutes to warm up before they started each session. In the training phrase, participants entered words freely using the corresponding keyboard. Participants rested for five minutes between sessions to avoid fatigue. In average, participant spent xx minute to complete the experiment.

%用户使用每一种实验设置之前,都有5分钟的热身时间,通过输入例句来热身。用户在切换实验设置的时候休息5分钟时间,以避免疲劳。用户在每个实验设置下需要完成与实验二相同的五个文本输入任务,每种实验设置下的实验时长大约为xx分钟。实验的总耗时为xx分钟。

\subsection{Result: TypeBoard vs. Regular Board}

我们首先讨论有/无触觉反馈对用户打字可用性、效率的影响,纹理landmark的问题会在稍后讨论。

我们采用了什么样的统计分析方法,对于xx等违反正态分布的量,我们通过xx方法来进行校正。如果一个独立变量对结果有显著性影响,我们采用xx方法来检验变量两两之间的显著性。

对用户行为、用户体验、输入效率的影响。

要注意的点:
1. 有无landmarks两种情况下,用户盲打的时间对比;每个按键点云标准差的对比;误触数量的对比;错误率。

\subsubsection{Completion Time}

English.

在誊写任务下可以对比打字速度。

\subsubsection{Error Rate (Text Entry)}

CER

UER

\subsubsection{Detected Unintentional Touches Percentage}

\subsubsection{Time Components}

\subsubsection{Subjective Rating and Feedback}

English.

物理负担(疲劳程度),心理负担,主观输入速度,主观输入准确率(误触准确率,而非选词准确率)。

将手指休息在键盘上的频次 over 任务,技术。

\subsubsection{Percentage of EyeFree time}

\subsection{Result: TypeBoard with/without tactile landmarks}

在上面的分析中我们已经看到,自然的防误触算法对用户打字的可用性、效率和用户体验都有好处。接下来,我们来讨论在防误触的情况下,有/无触觉landmarks对文本输入可用性的影响。

对用户行为、用户体验、输入效率的影响。

\subsection{Discussion}
