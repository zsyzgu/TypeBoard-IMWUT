\section{Study 3: Evaluation on TypeBoard}

实验三的目的有两点:第一,评测TypeBoard的性能,包括在不同的文本输入任务下的输入效率和主观用户体验,baseline是没有防误触算法的触屏键盘。第二,相关工作中我们提到,在防误触触屏键盘上,有望通过加上纹理反馈帮助用户盲打,提高输入效率,本实验探索了这一假设是否成立。

\subsection{Participants}

我们从本地的校园中邀请了15名用户参与实验,他们的年龄从xx岁到xx岁不等,平均数是xx,标准差是xx,其中有xx名女性。所有的用户都是右撇子,所有用户有超过xx年的手机文本输入经历,xx名用户常用平板电脑进行文本输入。这些用户没有参与过前两个实验。

\subsection{Design and Procedure}

我们采取了within-subject的实验方法,对比了三种不同的实验设置(如图xx所示):

\begin{enumerate}
	\item{\textbf{Config. 1):} \emph{Regular Soft Keyboard.} 没有防误触功能的软键盘。}
	\item{\textbf{Config. 2):} \emph{TypeBoard.} TypeBoard,有防误触功能的软键盘,这也是实验二的设置。}
	\item{\textbf{Config. 3):} \emph{TypeBoard + Tactile Landmarks.} 在TypeBoard的基础上,在每个按键的位置上贴上了0.05毫米厚的贴纸,其中F键和J键上各再加上了一条0.05毫秒厚的横杠,用于模拟物理键盘上的触觉Lankmakrs。0.05毫米的厚度可以让用户明显地摸到每个键的边缘,同时符合可变形触摸屏的工艺要求[xx]。}
\end{enumerate}

每名用户使用这三种实验设置来完成五个文本输入任务,除了前两个实验中所述的四个任务(填写个人信息、描述个人爱好、模拟开卷考试、看图写话)以外,实验三中新增了文本誊写任务,具体是誊写一个中文句子5次(某中文打字测速网站的第一句话)。文本誊写任务在文本输入相关工作中很常见,一般用于测试文本输入法的输入效率上限[xx,xx,xx,xx]。为了平衡学习效应,我们采用拉丁方来规定用户使用这三种实验设置的顺序,和完成五个输入任务的顺序。

【图:实验三要对比的三种实验设置】

用户使用每一种实验设置之前,都有5分钟的热身时间,通过输入例句来热身。用户在切换实验设置的时候休息5分钟时间,以避免疲劳。用户在每个实验设置下需要完成与实验二相同的五个文本输入任务,每种实验设置下的实验时长大约为xx分钟。实验的总耗时为xx分钟。

\subsection{Appartus}

除了触摸屏存在三种不同的设置以外,其余设备与实验二完全一样。桌面上只有触摸板,鼠标,显示器和耳机这些设备,用户通过耳机来获取打字瞬间的声音反馈。

\subsection{Result: TypeBoard vs. Regular Board}

我们首先讨论有/无触觉反馈对用户打字可用性、效率的影响,纹理landmark的问题会在稍后讨论。

我们采用了什么样的统计分析方法,对于xx等违反正态分布的量,我们通过xx方法来进行校正。如果一个独立变量对结果有显著性影响,我们采用xx方法来检验变量两两之间的显著性。

对用户行为、用户体验、输入效率的影响。

要注意的点:
1. 有无landmarks两种情况下,用户盲打的时间对比;每个按键点云标准差的对比;误触数量的对比;错误率。

\subsubsection{Completion Time}

English.

在誊写任务下可以对比打字速度。

\subsubsection{Error Rate (Text Entry)}

CER

UER

\subsubsection{Detected Unintentional Touches Percentage}

\subsubsection{Time Components}

\subsubsection{Subjective Rating and Feedback}

English.

物理负担(疲劳程度),心理负担,主观输入速度,主观输入准确率(误触准确率,而非选词准确率)。

将手指休息在键盘上的频次 over 任务,技术。

\subsubsection{Percentage of EyeFree time}

\subsection{Result: TypeBoard with/without tactile feedback}

在上面的分析中我们已经看到,自然的防误触算法对用户打字的可用性、效率和用户体验都有好处。接下来,我们来讨论在防误触的情况下,有/无触觉landmarks对文本输入可用性的影响。

对用户行为、用户体验、输入效率的影响。

\subsection{Discussion}
