\section{Study 3: Evaluation on TypeBoard}

实验三的目的有两点:第一,评测TypeBoard的性能,包括在不同的文本输入任务下的输入效率和主观用户体验,baseline是没有防误触算法的触屏键盘。第二,相关工作中我们提到,在防误触触屏键盘上,有望通过加上纹理反馈帮助用户盲打,提高输入效率,本实验探索了这一假设是否成立。

\subsection{Design and Procedure}

我们从本地的校园中邀请了15名用户参与实验,他们的年龄从xx岁到xx岁不等,平均数是xx,标准差是xx,其中有xx名女性。所有的用户都是右撇子,所有用户有超过xx年的手机文本输入经历,xx名用户常用平板电脑进行文本输入。这些用户没有参与过前两个实验。

我们采取了within-subject的实验方法,对比了三种不同的实验设置(如图xx所示):第一种设置,普通的触摸屏键盘;第二种设置,TypeBoard算法+整张的Qwerty布局贴纸,这也是实验二的设置;第三种设置,TypeBoard算法+沿着每个按键模切的Qwerty布局贴纸,其中F键和J键上和物理键盘上一样有额外的触觉反馈。贴纸的厚度是xx毫米,用户可以摸到每个键的边缘。每名用户依次使用这三种实验设置来完成前两个实验中提到的五种文本输入任务。为了平衡学习效应,我们采用拉丁方来规定用户使用这三种实验设置的顺序。

【图:实验三要对比的三种实验设置】

用户使用每一种实验设置之前,都用5分钟的热身时间,通过输入例句来热身。用户在切换实验设置的时候休息5分钟时间,以避免疲劳。用户在每个实验设置下需要完成与实验二相同的五个文本输入任务,每种实验设置下的实验时长大约为xx分钟。实验的总耗时为xx分钟。

\subsection{Appartus}

除了触摸屏存在三种不同的设置以外,其余设备与实验二完全一样。桌面上只有触摸板和显示器这两种设备,用户通过耳机来获取打字瞬间的声音反馈。

\subsection{Result}

要注意的点:
1. 有无landmarks两种情况下,用户盲打的时间对比;每个按键点云标准差的对比;误触数量的对比;错误率。

我们采用了什么样的统计分析方法,对于xx等违反正态分布的量,我们通过xx方法来进行校正。如果一个独立变量对结果有显著性影响,我们采用xx方法来检验变量两两之间的显著性。

\subsubsection{Completion Time}

English.

在誊写任务下可以对比打字速度。

\subsubsection{Error Rate (Text Entry)}

CER

UER

\subsubsection{Detected Unintentional Touches Percentage}

\subsubsection{Time Components}

\subsubsection{Subjective Rating and Feedback}

\subsubsection{Percentage of EyeFree time}

\subsection{Discussion}

\subsubsection{TypeBoard vs. Regular Board}

English.

对用户行为、用户体验、输入效率的影响。

\subsubsection{TypeBoard with/without tactile feedback}

English.

对用户行为、用户体验、输入效率的影响。
