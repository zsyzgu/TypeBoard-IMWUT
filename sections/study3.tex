\section{Study 3: Evaluation on TypeBoard}

实验三的目的有两点:第一,评测TypeBoard的性能,包括在不同的文本输入任务下的输入效率和主观用户体验,作为比较的baseline是没有防误触算法的触屏键盘。第二,由于TypeBoard具有防误触的能力,我们猜测此时在触屏键盘上加上纹理反馈能帮助用户实现盲打,提高输入效率,本实验同时探索了这一假设是否成立。

\subsection{Design and Procedure}

我们从本地的校园中邀请了16名用户参与实验,他们的年龄从xx岁到xx岁不等,平均数是xx,标准差是xx,其中有xx名女性。所有的用户都是右撇子,所有用户有超过xx年的手机文本输入经历,xx名用户常用平板电脑进行文本输入。这些用户没有参与过前两个实验。

我们采取了within-subject的实验方法,对比了三种不同的实验设置(如图xx所示):第一种设置,普通的触摸屏键盘,触摸即打字;第二种设置,TypeBoard算法+整张的Qwerty布局贴纸;第三种设置,TypeBoard算法+沿着每个按键模切的Qwerty布局贴纸,其中F键和J键上和物理键盘上一样有额外的触觉反馈,贴纸的厚度是xx毫米,用户可以摸到每个键的边缘。每名用户依次使用这三种实验设置来完成前两个实验中提到的五种文本输入任务。为了平衡学习效应,我们采用拉丁方来规定用户使用这三种实验设置的顺序。

【图:实验三要对比的三种实验设置】

用户使用每一种实验设置之前,都用5分钟的热身时间,通过输入例句来热身。用户在切换实验设置的时候休息5分钟时间,以避免疲劳。用户真正花在文本输入任务实验上的实验是xx分钟每五个输入任务。实验的总耗时为xx分钟。

\subsection{Appartus}

与实验二非常类似,简述。

\subsection{Result}

我们采用了什么样的统计分析方法,对于xx等违反正态分布的量,我们通过xx方法来进行校正。如果一个独立变量对结果有显著性影响,我们采用xx方法来检验变量两两之间的显著性。

\subsubsection{Speed}

\subsubsection{Error Rate (Text Entry)}

\subsubsection{Detected Unintentional Touches Percentage}

\subsubsection{Time Components}

\subsubsection{Subjective Rating and Feedback}

\subsection{Discussion}

\subsubsection{TypeBoard vs. Regular Board}

English.

对用户行为、用户体验、输入效率的影响。

\subsubsection{TypeBoard with/without tactile feedback}

English.

对用户行为、用户体验、输入效率的影响。
