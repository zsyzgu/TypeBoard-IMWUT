\section{introduction}

目前,手机已经成为了最成功的便携式设备[xx],手机上的触摸输入也代替个人电脑的键盘和鼠标成为最常用的输入方式。然而,在办公的场景下,经常需要完成大量的文本输入任务,此时物理键盘与触屏键盘相比仍然有着无可比拟的优势。先前的实验表面,普通用户在物理键盘上的平均打字速度为xx.x-WPM,而在触屏键盘上只有xx.x-WPM。xxx指出,物理键盘上的按键有三种状态,分别是released,touched和pressed,而触屏键盘上的按键只有released和pressed两个状态,而物理键盘上多出来的touched状态是其之所以高效的重要原因[xx]。首先,用户可以通过touched状态aligh自己的fingers,从而完成盲打;第二,从touched到pressed这个过程的物理反馈给用户提供了更强的确认感,对输入效率有着微弱而显著的帮助;第三,用户可以将手指休息在键帽上,从而一定程度上避免疲劳。

为了弥补触屏键盘上touched状态的缺失问题,我们提出了TypeBoard,一个基于压敏触屏键盘的防误触算法,能够区分打字触摸和非有意触摸。如此一来,我们就可以将打字触摸视为物理键盘上的pressed状态,将其它触摸理解为touched状态。在正确区分打字触摸和非有意触摸的前提下,先前工作有不少方法可用于补足触屏键盘和物理键盘之间的gap,例如,触屏键盘上可以加上可变形的纹理,来帮助用户align-fingers[xx],也可以给每次点击加上震动反馈[xx]和声音反馈[xx]。

在触屏键盘上区分打字触摸和非有意触摸并不是我们的首创。在2013年,TapBoard就曾经提出将快速的Tapping动作看作打字,而将其它触摸事件视为误触。TapBoard通过触摸时间和位移的阈值来判断误触,然后以一个誊写的实验证明了用户可以适应该输入方式,且不影响用户在触屏上打字的输入效率。这一方法有不少缺陷:首先,仅通过阈值方法来防误触的准确率不高;第二,由于TapBoard将触摸时间作为误触的判据,它只能在手指released时做出判断,而正常的键盘会在pressed以后立刻做出判断,这会影响输入效率和用户体验[xx];第三,触屏键盘的防误触能力会影响用户的行为模式,比如,用户在防误触能力强的触屏上打字时,可能会更自然、更频繁地将手指休息在键盘上。由于TapBoard先设立了规则,再让用户去适应,因此无法探索用户在自然状态下的输入行为。

在这份工作中,我们希望探索用户在完美防误触触屏键盘上最自然的打字行为,并根据用户的行为模式,来设计一款完美的防误触触屏键盘。用户行为和防误触算法是一个chicken-and-egg-conundrum,因此我们采用迭代的方式来求解这个问题。在防误触算法的帮助下,我们可以给用户提供纹理反馈和触觉反馈,我们还感兴趣于触屏键盘上这两种反馈对用户打字行为的影响。为了解决解答这些问题,我们组织了三个用户实验,分别用于回答以下三个研究问题。

\begin{enumerate}
	\item{\textbf{RQ1):} \emph{用户在一个想象中可以防误触的触屏键盘上的打字行为如何?} 我们组织实验一收集了用户在一个没有反馈的触摸板上打字的数据,用户在打字时想象该键盘能够完美地防止误触。每个用户在四种真实场景下完成文本输入任务。由于触摸板没有任何的反馈,用户不能真的打字,而是想象字母上屏了。在用户完成输入任务后,她需要标注触摸板的每次报点,标注过程中可以调出实验录屏作为参考。实验一共采集了xx个数据点,其中误触点占xx\%,远大于我们在正常触屏上的误触数量。实验一是TypeBoard算法的\textbf{first-iteration},我们开发了一个机器学习的方法来区分打字触摸和误触,准确率达到了xx.x\%,识别延迟时点击后xx毫秒。作为比较,先前工作中基于触摸时间和距离阈值的方法[xx]在该数据集上的准确率仅为xx.x\%,且只能在手指抬起时识别。我们针对fail-cases人工分析了错误原因,总结出用户打字时误触的行为规律,并针对这些规律优化了机器学习提取的特征。}
	\item{\textbf{RQ2):} \emph{在触屏键盘有防误触能力的情况下,纹理反馈或触觉反馈会对用户的打字行为造成影响吗?}我们组织了实验二。xxx。实验二同时是TypeBoard算法的\textbf{second-iteration}。xxx。}
	\item{\textbf{RQ3):} \emph{与普通的触屏键盘相比,防误触触屏键盘的交互效率和交互体验如何?}我们组织了实验三。我们在写日记、填问卷、誊写等多个输入任务下对比了TypeBoard和普通触屏键盘,我们发现TypeBoard在写日记、填问卷等任务下显著提升了用户体验,降低了疲劳,而在誊写任务下和普通触屏键盘没有差异。实验三同时是TypeBoard算法的\textbf{third-iteration},我们通过仿真发现,TypeBoard的最终版本在实验三的数据集(用户行为已经足够自然)下的准确率是xx.x\%,远高于baseline的xx.x\%。}
\end{enumerate}

这份工作有三个贡献点:第一,TypeBoard准确地、低延迟地区分了触屏打字时的typing和误触。第二,在TypeBoard的支持下,我们理解并总结了用户在“近乎完美的防误触触屏键盘”上的输入行为,也公开了该数据集。第三,我们通过评测实验证明,TypeBoard与传统触屏键盘相比,提高了输入效率,降低了用户疲劳程度,提高了用户体验。另外,我们还讨论了人机交互研究领域中存在的一个普遍的问题,即在很多的工作中用户的行为和技术实现是相互影响的[xx],而大部分的先前工作都忽略了这一效应的存在,因此它值得受到更多的关注。
