\section{introduction}

目前,手机已经成为了最成功的便携式设备[xx],手机上的触摸输入也代替个人电脑的键盘和鼠标成为最常用的输入方式。然而,在办公的场景下,经常需要完成大量的文本输入任务,此时物理键盘与触屏键盘相比仍然有着无可比拟的优势。先前的实验表面,普通用户在物理键盘上的平均打字速度为xx.x-WPM,而在触屏键盘上只有xx.x-WPM。xxx指出,物理键盘上的按键有三种状态,分别是released,touched和pressed,而触屏键盘上的按键只有released和pressed两个状态,而物理键盘上多出来的touched状态是其之所以高效的重要原因[xx]。首先,用户可以通过touched状态aligh自己的fingers,从而完成盲打;第二,从touched到pressed这个过程的物理反馈给用户提供了更强的确认感,对输入效率有着微弱而显著的帮助;第三,用户可以将手指休息在键帽上,从而一定程度上避免疲劳。

为了弥补触屏键盘上touched状态的缺失问题,我们提出了TypeBoard,一个基于压敏触屏键盘的防误触算法,能够区分打字触摸和非有意触摸。如此一来,我们就可以将打字触摸视为物理键盘上的pressed状态,将其它触摸理解为touched状态。在正确区分打字触摸和非有意触摸的前提下,先前工作有不少方法可用于补足触屏键盘和物理键盘之间的gap,例如,触屏键盘上可以加上可变形的纹理,来帮助用户align-fingers[xx]。

在触屏键盘上区分打字触摸和非有意触摸并不是我们的首创。在2013年,TapBoard就曾经提出将快速的Tapping动作看作打字,而将其它触摸事件视为误触。TapBoard通过触摸时间和位移的阈值来判断误触,然后以一个誊写的实验证明了用户可以适应该输入方式,且不影响用户在触屏上打字的输入效率。这一方法有不少缺陷:首先,仅通过阈值方法来防误触的准确率不高;第二,由于TapBoard将触摸时间作为误触的判据,它只能在手指released时做出判断,而正常的键盘会在pressed以后立刻做出判断,这会影响输入效率和用户体验[xx];第三,触屏键盘的防误触能力会影响用户的行为模式,比如,用户在防误触能力强的触屏上打字时,可能会更自然、更频繁地将手指休息在键盘上。由于TapBoard先设立了规则,再让用户去适应,因此无法探索用户在自然状态下的输入行为。

在这份工作中,我们希望探索用户在防误触触屏键盘上自然的打字行为,并根据用户的行为模式,来设计一款鲁棒的防误触触屏键盘。用户行为和防误触算法是一个chicken-and-egg-conundrum,因此我们采用迭代的方式来求解防误触算法。最后,我们希望评测防误触算法在体验和效率上对触屏打字的改进。为了解决解答这些问题,我们组织了三个用户实验,分别用于回答以下三个研究问题。

\begin{enumerate}
	\item{\textbf{RQ1):} \emph{用户在一个想象中的防误触触屏键盘上的打字行为如何?}我们组织实验一收集了用户在一个没有反馈的触摸板上打字的数据,用户在打字时想象该键盘能够完美地防止误触。每个用户在四种真实场景下完成文本输入任务。由于触摸板没有任何的反馈,用户不能真的打字,而是想象字母上屏了。在用户完成输入任务后,她需要标注触摸板的每次报点,标注过程中可以调出实验录屏作为参考。实验一共采集了xx个数据点,其中误触点占xx\%,远大于我们在正常触屏上的误触数量。实验一是TypeBoard算法的\textbf{first-iteration},我们开发了一个机器学习的方法来区分打字触摸和误触,准确率达到了xx.x\%,识别延迟时点击后xx毫秒。作为比较,先前工作中基于触摸时间和距离阈值的方法[xx]在该数据集上的准确率仅为xx.x\%,且只能在手指抬起时识别。我们针对fail-cases人工分析了错误原因,总结出用户打字时误触的行为规律,并针对这些规律优化了机器学习提取的特征。}
	\item{\textbf{RQ2):} \emph{用户在防误触触屏键盘上的打字行为如何?}在实验一以后,我们已经得到了初版的TypeBoard防误触算法,我们组织实验二收集了用户在该防误触键盘上打字的数据,键盘会对用户的打字触摸给出反馈(声音反馈+字母上屏),而不对误触作出反馈。用户同样在四种真实场景下完成文本输入任务,在每个用户完成实验以后,我们都会更新数据集,重新训练防误触算法,从而迭代地去实现一个鲁棒地防误触算法。由于实验二的标注难度降低(用户只需标注机器学习判错的数据点),我们采集了比实验一更多的xx个数据点。其中误触点占xx\%,显著高于实验一的误触点数量,这说明真正的防误触算法会提高用户对触屏的信任感,使误触点数量增加。在该数据集上,我们的算法的识别准确率达到了xx.x\%,进一步拉大了与baseline(xx.x\%)的差距。与实验一一样,我们针对fail-cases人工分析了错误原因,新增了用户打字时误触的行为规律,也更新了机器学习提取的特征。}
	\item{\textbf{RQ3):} \emph{防误触键盘对用户体验和效率有和影响?防误触键盘上加上纹理反馈有何影响?} 我们组织了实验三,在多个输入任务下对比了TypeBoard和普通触屏键盘的用户体验和输入效率,我们发现TypeBoard在写日记、填问卷等任务下显著提升了用户体验,降低了疲劳,而在誊写任务下和普通触屏键盘没有差异。实验三同时对比了在有无纹理反馈两种设置下用户的行为规律,和TypeBoard性能变化,结果发现,纹理反馈使得用户更频繁地将手指放置在触摸屏上,误触点数大大增加,在这一极端的数据集下,我们的算法准确率为xx.x\%,而baseline的准确率仅为xx.x\%。纹理反馈大大提升了用户的输入效率,观察实验视频我们发现,这一提升可能是因为防误触算法+纹理反馈合力使能了用户在触屏键盘上的盲打。}
\end{enumerate}

这份工作有三个贡献点:第一,TypeBoard准确地、低延迟地区分了触屏打字时的typing和误触。第二,在TypeBoard的支持下,我们理解并总结了用户在防误触触屏键盘上的输入行为,我们公开了该数据集。第三,我们通过评测实验证明,TypeBoard与传统触屏键盘相比,提高了输入效率,降低了用户疲劳程度,提高了用户体验。此外,我们还讨论了人机交互研究领域中存在的一个普遍的问题,即用户行为和技术实现之间相互影响的现象是广泛存在的,而大部分的先前工作都忽略了这一效应的存在[xx],它值得更多的关注。
