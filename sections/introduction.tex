\section{introduction}

随着平板电脑的普及[xx],在平板电脑上进行文本输入的需求也在增长。平板电脑用户需要使用文本输入来完成搜索、笔记等简单的功能,有的用户甚至需要用平板电脑办公。触屏软键盘是平板电脑上通用的文本输入方案,然而,软键盘在输入效率(xx-WPM[xx] vs. xx-WPM[xx])、疲劳程度[xx]和视觉注意力负担[xx]等诸多方面上与和物理键盘存在巨大的差异。在物理键盘上打字时,用户可以将手指休息在键盘上,这有两大优势:第一,用户可以将手指休息在键盘上,减轻疲劳;第二,用户可以通过触摸按键纹理来对齐他们的手指,从而实现盲打,降低视觉注意力的占用,大大提高输入效率。目前,用户不能将手指休息在软键盘上,因为这会导致误触。在篇论文中,我们提出了TypeBoard,一款软键盘上的防误触算法,使得用户可以将手指休息在触屏上。在TypeBoard的帮助下,我们有机会通过添加膜层[xx]或可变形触屏[xx]等方案给软键盘添加触觉纹理,从而支持软键盘上的盲打,弥补软硬键盘之间的差距。

在触屏键盘上区分打字点击和“误触”并不是我们的首创。在2013年,TapBoard就曾经提出将Tapping动作看作打字点击,而将其它触摸事件视为误触。Tapping的定义是“触摸时间低于xx毫秒,位移低于xx毫米的点击”。用户需要主动去适应TapBoard的技术方案,这存在着准确率低、影响用户自然性和舒适性等等诸多问题,我们会在相关工作中详细描述。为了克服TapBoard的局限性,在这份工作中我们从用户的角度出发,将打字点击定义为“表达键入意图的点击”。特别地,我们希望理解用户在防误触触屏键盘上自然的打字行为,并根据用户的行为模式,来设计一款鲁棒的防误触触屏键盘。

“理解用户在防误触触屏键盘上自然的打字行为”是困难的,这涉及到人机交互研究领域中一个广泛存在的问题,即交互技术和用户行为之间可能是互相影响的。这一现象在防误触算法的设计上尤为突出:一方面,防误触算法的能力会显著影响用户的行为,举例而言,用户在防误触能力强的软键盘上会将手指休息在触屏上,带来更多更有挑战性的误触点;另一方面,我们对用户行为的理解又能帮助防误触算法的设计。然而,几乎所有相关论文都忽略了这一现象,有的论文在收集用户行为数据时没有给出反馈[xx,xx,xx],其它论文给出了替代性的反馈[xx,xx,xx]。为了弥补这个缺陷,我们采用迭代的方式来设计TypeBoard,逐步理解用户的自然打字行为、设计与行为相符的防误触算法。我们共组织了三个用户实验,分别回答了以下三个研究问题:

\begin{enumerate}
	\item{\textbf{RQ1):} \emph{用户在一个想象中的防误触触屏键盘上的打字行为如何?}我们组织实验一收集了用户在无反馈触摸板上打字的数据,用户想象该键盘能够完美地防止误触。由于触摸板没有反馈,用户不能真的打字,而是想象字母上屏了。用户完成了多种文本输入任务,并人工标注了数据。实验一共采集了xx个数据点,其中误触点占xx\%,远大于我们在正常触屏上的误触数量。我们分析了用户的行为,并开发了针对性算法,识别误触的准确率达到了xx.x\%,延迟为手指落下后xx毫秒。作为比较,TapBoard[xx]在该数据集上的准确率仅为xx.x\%,且只能在手指抬起时识别。}
	\item{\textbf{RQ2):} \emph{用户在防误触触屏键盘上的打字行为如何?}在实验一以后,我们得到了初步的防误触算法,我们组织实验二收集了用户在该防误触键盘上打字的数据。在实验二中,用户能够键入字母,同时得到声音反馈。用户完成了多种文本输入任务,并人工标注了数据。共有25名用户参与了实验二,他们被分成了五组。在每五个用户完成实验之后,实验者都会更新数据集,添加必要的特征向量,重新训练防误触算法。最终,实验二共采集到xx个数据点,其中误触误触点占xx\%,显著高于实验一的误触点数量,这暗示真正的防误触算法会提高用户对触屏的信任感。在该数据集上,我们的算法的识别准确率达到了xx.x\%,进一步拉大了与TaoBoard(xx.x\%)的差距。}
	\item{\textbf{RQ3):} \emph{防误触键盘对用户体验和效率有和影响?防误触键盘上加上纹路有何影响?} 我们组织了实验三,在多个输入任务下对比了TypeBoard和普通触屏键盘的用户体验和输入效率。结果表明,TypeBoard在写日记、填问卷等任务下显著提升了用户体验,降低了疲劳,而在誊写任务下和普通触屏键盘没有差异。实验三同时对比了在有无纹理反馈两种设置下用户的行为规律,和TypeBoard性能变化,结果发现,纹理反馈使得用户更频繁地将手指放置在触摸屏上,误触点数大大增加,在这一极端的数据集下,我们的算法准确率为xx.x\%,而baseline的准确率仅为xx.x\%。纹理反馈大大提升了用户的输入效率,观察实验视频我们发现,这一提升可能是因为防误触算法+纹理反馈合力使能了用户在触屏键盘上的盲打。}
\end{enumerate}

这份工作有三个贡献点:第一,TypeBoard准确地、低延迟地区分了触屏打字时的typing和误触。第二,在TypeBoard的支持下,我们理解并总结了用户在防误触触屏键盘上的输入行为,我们公开了该数据集。第三,我们通过评测实验证明,TypeBoard与传统触屏键盘相比,提高了输入效率,降低了用户疲劳程度。
