\section{introduction}

物理键盘上的按键有三种状态,分别是released,touched和pressed,而触屏键盘上的按键只有released和pressed两个状态。Touched状态的缺失是触屏输入法不如物理键盘的重要原因,原因有三点:首先,用户可以通过触摸物理按键上的纹理来align他们的手指;第二,用户可以将手指休息在物理键盘上。为了弥补触屏键盘上touched状态的缺失问题,我们提出了TypeBoard,一个在压敏触屏键盘上的防误触算法,能够区分type-touches和unintentional-touches,判断用户的触摸意图。TypeBoard使得用户可以在键盘上休息手指,也为相关工作中在触屏上加上纹理[xx]和触觉反馈[xx]的构想提供了可行性。

在触碰键盘上区分Type-Touches和Unintentional-Touches并不是我们的首创。在2013年,TapBoard就曾经提出将Tapping动作看作打字,而将其它较轻的触摸事件视为误触,算法上使用了时间和位移的阈值在区分打字和误触。TapBoard先设计了基于阈值的防误触算法,再证明用户可以适应这个算法。这个研究方法有三个缺陷:一是仅通过阈值方法来防误触的准确率不高;二是正常键盘会在pressed以后即作出判断,而TapBoard只能在released时判断,这会影响用户体验和输入效率[xx]。第三,也是最重要的是,键盘的防误触能力会影响用户的心理模型,比如,用户在防误触能力很强的触屏键盘上,可能更倾向于把手指休息在键盘上。TapBoard无法理解用户在防误触能力很强的触屏上的输入行为。另外,TapBoard在实验设计上有一个错误,他们通过一个基于誊写任务的对比实验,证明了TapBoard的输入效率和传统触屏键盘无异,这是以偏概全的。我们通过Pilot-study发现,在誊写等快速打字的任务中,用户很少将手指休息在触屏上,因此TapBoard无法证明用户在写日记等任务中不会因为手指的休息造成误触问题。

我们提出了TypeBoard,我们的目标是窥探用户在一个“完美的防误触触屏键盘”上打字的行为,并根据用户的心理模型,来设计一款能完美防误触的触屏键盘。由于防误触能力会影响用户的心理模型,我们采用了迭代的方法来逼近“完美的防误触键盘”及其上的用户行为,实现方法是迭代地采集用户数据和使用机器学习来优化防误触算法。我们设计实验一采集了用户在一块没有任何反馈的普通触屏上“打字”的数据,我们要求用户在输入过程中想象该触屏能够“完美防误触”,实验后让用户标注每个触点是否有意,我们通过机器学习设计了初版的TypeBoard算法。接着,我们设计实验二采集了用户在初版TypeBoard上的文本输入数据,TypeBoard根据自己的预测给出了视觉和听觉反馈,我们更新数据集以及改进了机器学习模型来设计了终版TypeBoard算法。实验二采集到的数据是一个严格的数据集(有很多非有意触碰点),TypeBooard的准确率达到了xx.x\%,预测时间是touch之后的xx-ms以内。相比之下,baseline-TapBoard的准确率只有xx.x\%,且只能在release时做出判断。最后,我们在写日记、填问卷、誊写等多个输入任务下对比了TypeBoard、TapBoard和普通触屏键盘,我们发现TypeBoard在写日记、填问卷等任务下显著提升了用户体验,降低了疲劳,而在誊写任务下和普通触屏键盘没有差异。

这份工作有三个贡献点:第一,TypeBoard准确地、低延迟地区分了触屏打字时的typing和误触。第二,在TypeBoard的支持下,我们理解并总结了用户在“近乎完美的防误触触屏键盘”上的输入行为,也公开了该数据集。第三,我们通过评测实验证明,TypeBoard与传统触屏键盘相比,提高了输入效率,降低了用户疲劳程度,提高了用户体验。

(可选)我们在实验方法的创新上有贡献,我们提倡采用迭代的方法来处理技术和心理模型会互相影响的情况, 并提倡用在鲁棒技术上采集的实验数据来做机器学习准确率的评测,因为这种情况下用户的数据更加苛刻。
