\section{Related Work}

\subsection{Unintentional Touch Rejection on Touchscreen}

Touch is the main input channel on touchscreen devices such as smartphone, tablet and tabletop, but not all contacts on the touchscreen are intended to trigger a digital response. Those touches that do not contribute to any interaction goal are known as unintentional touches \cite{2020-TabletopTouch}, or namely accidental/unwanted touches \cite{2015-GestureOn,2012-IdentifyUnint}. Since unintentional touches trigger unwanted interaction, the user's on-going behavior will be interrupted by the unintentional touch [xx]. Moreover, the user needs to spend extra time to cancel the accidentally triggered response [xx], which affects the efficiency [xx] and naturalness [xx] of the interaction \cite{2014-PenMightier, 2020-TabletopTouch}.

However, unintentional touch is inevitable in touchscreen interaction. For example, the thenar eminence on the human hand will constantly contact the touchscreen during the daily use of smartphones [xx]. Fortunately, we can identify and filter out these unintentional touches by software techniques. In the literature, the methods of preventing unintentional touches has been extensively studied. We compare existing work in two aspects: (1) the use scenario, and (2) the sensor.

\subsubsection{Unintentional Touch Rejection over use scenarios}

The definition of unintentional touch varies in different use scenarios. When the application is not limited, unintentional touches refer to those touches that do not contribute to any interaction goal \cite{2020-TabletopTouch}. The boundary of intentional and unintentional touches will be more clear in specific task. For examples, in the text entry task, unintentional touches are those touches that do not intended to entry words [xx].

A few studies \cite{2012-IdentifyUnint,2020-TabletopTouch} and a large amount of patents \cite{2016-Classification,2006-PadUnint,2013-System,2013-Precluding,2015-TouchScreen} attempt to identify unintentional touch over applications. Metero et al. presented guidelines to reduce the amount of unintentional touches on smartphones \cite{2012-IdentifyUnint}. Their filtering criteria rejected 79.6\% of unintentional touches. Xu et al. identify and filter out unintentional touches on interactive tabletop using gaze direction, head orientation and screen contact data \cite{2020-TabletopTouch}. The accuracy was 91.3\%. These approaches suffered from low recognition rate.

In the literature, more studies were conducted to identify unintentional touch in specific scenarios. TapBoard and TapBoard2 discussed this issue in text entry tasks \cite{2013-TapBoard,2016-TapBoard2}. TapBoard regarded short-term tapping actions as keystrokes and other contacts as unintentional touches. The system reported a keystroke when the touch duration is shorter than 450 ms and the touch movement is shorter than 15 mm. Users adapted their behaviors to these thresholds, so that TapBoard achieved an accuracy of roughly 97\%. Based on TapBoard, TapBoard2 was able to disambiguate typing and pointing actions with an accuracy of greater than 95\%.
While inking on tablets, unintentional touch resulted in a great effect on user behavior and was one of the most prominent features identified as problematic by participants \cite{2014-PenMightier}, e.g., users were forced to write in an uncomfortable position to avoid the ‘palm touch’ screen. Several studies were proposed to reject unintentional touch in pen and tablet interaction \cite{2013-PalmInput,2014-PenUnint,2014-PalmRejection}. Schwarz et al. achieved the best performance \cite{2014-PalmRejection} by leveraging spatiotemporal touch features, reducing accidental palm inputs to 0.016 per pen stroke, while correctly passing 98\% of stylus inputs.
In smartphone interaction, palm touches are often considered as unintentional touches. PalmTouch used these "unintentional touches" as intentional input methods \cite{2018-PalmTouch}, such as a shortcut. PalmTouch differentiated between finger and palm touch with an accuracy of 99.53\% in realistic scenarios. GestureOn enabled gesture shortcuts in the standby mode by which a user can draw a gesture on the touchscreen before the screen is turned on \cite{2015-GestureOn}. GestureOn acquired 98.2\% precision and 97.6\% recall on detecting gestures from accidental touches.
The studies above explored the feasibility of rejecting unintentional touch in specific scenarios. Their performances were generally high.

%我们可以通过对误触的定义来分类相关工作种触屏上的防误触算法。在广义的交互场景中,误触包括所有不代表用户输入意图的触摸点击\cite{2020-TabletopTouch};而在特定的任务场景中,误触的定义会更加明确,比如在触摸屏的打字任务中,误触指的是所有不代表打字行为的触摸点击[xx];又比如在写字板的任务中,误触指的是所有非写字笔导致的触摸点击[xx]。

%将unintentional touch定义为不代表任何输入意图的触摸点\cite{2020-TabletopTouch},这一类工作研究的是设备上通用的防误触问题,不区分具体的任务。也有工作将误触称为unwanted touch或accidental touch\cite{2015-GestureOn,2012-IdentifyUnint}。这一类工作的优点在于适用于广泛的应用场景和任务,其tradeoff是相对降低的识别准确率,例如手机上利用触屏报点识别误触的recall只有76.6\%\cite{2012-IdentifyUnint},又如在大桌面上结合了触屏报点、头动、眼动等信息判断误触,其准去率(F1 score)也只有91.3\%。这一类工作有两大致命的缺点,一是准确率很低,这一低准确率可能暗示着正常点击和“不代表任何输入意图的点击”在现有的传感器数据下有时候是不可区分的,而不会future-work中所说的随着机器学习的进步而得到一个可用的水平;二是他们无法通过简短的几个lab-study证明算法的普适性,而只能在作为limitation进行讨论。

%在学术界,更多的工作研究了限定的场景和任务下的防误触问题。有的工作专门讨论了触屏键盘的防误触问题\cite{2013-TapBoard,2016-TapBoard2},在这一类工作中,误触的定义是除了“有意的打字点击”以外的触摸点。TapBoard\cite{2013-TapBoard}将短暂的tap动作视为有意的打字点击,而将其它触摸事件视为误触点,并通过阈值的方法区分它们,结果是用户可以适应这种技术,在TapBoard上打字的效率与在普通触屏上打字无异,而用户平时可以将手指轻轻放在屏幕上休息。TapBoard2\cite{2016-TapBoard2}我还没有看明白呢!有的工作着力于区分触屏上的电容笔点击和手掌的误触,基于报点和电容屏的数据,这些工作采用时空特征和简单的机器学习,达到了xx.x\%以上的准确率,具体也是还没看呢!\cite{2013-PalmInput,2014-PenUnint,2014-PalmRejection}。还有的工作在防误触的情况下识别了特定的触摸手势。例如PalmTouch\cite{2018-PalmTouch}以99.5\%的准确率区分了手指输入和手掌输入,从而将手掌输入作为特定功能的快捷方式。GestureOn\cite{2015-GestureOn}以97.9的准确率(F1 score)区分了黑屏时的手势输入和日常生活中的误触,也是为了表达快捷方式。以上这一类特定场景下的误触识别问题,其准确率往往比较高,更接近实用的水平,作为tradeoff是仅限于特定的应用场景和任务。

\subsubsection{Unintentional Touch Rejection over sensors}

A mass of studies have been conducted to recognize unintentional input on touchscreen devices, including smartphones \cite{2012-IdentifyUnint,2015-GestureOn,2018-PalmTouch,2019-BeyondUnint}, tablets \cite{2006-PadUnint,2014-PenUnint,2014-PalmRejection,2013-TapBoard,2016-TapBoard2} and tabletops \cite{2020-TabletopTouch}.
Most techniques leverage spatiotemporal features of touchpoints \cite{2012-IdentifyUnint,2006-PadUnint,2014-PalmRejection,2013-TapBoard,2016-TapBoard2} and capacitive images to identify unintentional touches \cite{2018-PalmTouch,2014-PenUnint}.
Metero et al. explored the feasibility of rejecting unintentional touch on smartphones using touchpoint patterns \cite{2012-IdentifyUnint}. They analyzed user behavior in three typical tasks including swipe interactions in the home view, phone call interaction, and general device handling. The authors proposed filtering criteria such as touch duration, position and trajectory pattern that rejected 79.6\% of unintentional touches whilst rejecting 0.8\% of intentional touches.
Schwarz el al. presented a probabilistic touch filtering approach that distinguish between legitimate stylus with palm touches on tablet computers \cite{2014-PalmRejection}. The method extracted features from touchpoints and used the decision forest model, reducing accidental palm inputs to 0.016 per pen stroke, while correctly passing 97.9\% of stylus inputs.
PalmTouch \cite{2018-PalmTouch} is an additional input modality that differentiates between touches of fingers and the palm. The intended palm touch supports different use cases, including the use as a shortcut and for improving reachability. PalmTouch used the raw capacitive image of the touchscreen as input and used Convolutional Neural Network (CNN) as the method, resulting in an accuracy of 99.53\% in realistic scenarios.
The above approaches rely on built-in sensors, and can be immediately applied to most of the existing smartphones and tablet computers. As a trade-off, they are limited by low recognition accuracy or fixed application scenarios.

Some related techniques enhanced the sensing ability to improve the performance of rejecting unintentional touches \cite{2015-GestureOn,2020-TabletopTouch,2001-PalmPressure,2019-BeyondUnint}.
GestureOn \cite{2015-GestureOn} distinguishes between intended gesture input with unintentional touches in the standby mode of smartphones. The user can trigger gesture shortcuts before the screen is turned on. GestureOn used most of the built-in sensors on the smartphone including proximity sensor, light sensor, IR sensors and Inertial Measurement Unit (IMU). The system also leverage the pressure associated with the touch event, which is not popularized on smartphones yet. Based on sensor fusion, GestureOn acquired 98.2\% precision and 97.6\% recall on detecting gestures from accidental touches.
Xu et al. leveraged gaze direction, head orientation and screen contact data to identify and filter out unintentional touches on interactive tabletop \cite{2020-TabletopTouch}. Result showed that the patterns of gaze direction and head orientation improved the accuracy of identifying unintentional touches by 4.3\%, reaching 91.3\%. The above approaches used additional sensors, and unsurprisingly improved the recognition rate.

%在智能手机\cite{2012-IdentifyUnint,2015-GestureOn,2018-PalmTouch,2019-BeyondUnint}、平板电脑\cite{2006-PadUnint,2014-PenUnint,2014-PalmRejection,2013-TapBoard,2016-TapBoard2}和大桌面\cite{2020-TabletopTouch}等等触摸设备上,都有大量防误触的相关工作,其中大多数工作都是仅利用传统触摸设备中自带的报点数据\cite{2012-IdentifyUnint,2006-PadUnint,2014-PalmRejection,2013-TapBoard,2016-TapBoard2}、电容屏图像\cite{2018-PalmTouch,2014-PenUnint}来判断误触的。Matero等人研究了利用智能手机触屏报点信息来识别误触的可行性\cite{2012-IdentifyUnint}。他们在智能手机上采集了17名用户在swipe interactions in the home view, traditional phone call interaction, and general device handling这三种典型的应用场景下的触屏报点数据,并提出了六个与点击时长、落点位置和轨迹规律相关的数值作为filtering criteria,最终能rejected 76.6\%的误触点,同时只rejected了0.8\%的有意点击。Schwarz等人xx\cite{2014-PalmRejection},提出了一个概率性的触摸过滤器用于区分legitimate stylus and palm touches。They used spatiotemporal features and decision forest model to distinguish palm touches from stylus input。他们的系统可以将accidental palm inputs降低至0.016每次pen stroke,同时正确地识别了97.9\%的笔迹输入。PalmTouch\cite{2018-PalmTouch}是一种附加的输入模态,which可以区分手指输入和手掌输入,并利用手掌输入作为快捷键来提高手机系统的可达性。PalmTouch将整个电容屏的图像信息作为输入,并用卷积神经网络训练二分类模型,最终达到了99.5\%的准确率。以上工作的优势在于仅利用了现呈的设备和传感器,其算法能够马上应用在目前大部分手机和平板电脑上;其缺点是低准确率或是受限的应用场景。

%另一部分工作在现有设备之上加入了其它的输入信道\cite{2015-GestureOn,2020-TabletopTouch,2001-PalmPressure,2019-BeyondUnint}。GestureOn\cite{2015-GestureOn}试图将黑屏时的手势输入和日常生活中的误触区分处理,这份工作广泛地利用了手机上所有能获取到的传感器数据,比如光敏传感器、接近光传感器、惯性传感器和带有压力的触屏报点信息。其中触屏上压力信息是技术上成熟,但还未完全普及的技术。融合了这些信息以后,GestureOn acquired 98.2\% precision and 97.6\% recall on detecting gestures from accidental touches。徐等人尝试解决大桌面上误触问题\cite{2020-TabletopTouch},它们将误触定义为一切不表达输入意图的触屏报点,在这份工作中,他们的防误触算法利用到了头动和眼动的信息,which在目前来说只能在实验室环境下获取,最后,这份工作达到了an F1 score of 91.3\%。以上的工作都在最常用的触屏设备上加上额外的传感器,能够不出意外地提升误触的识别准确率,tradeoff是不能马上应用在未经改装的手机上。

% Smartphone:`
% [2012-IdentifyUnint] 报点
% [2015-GestureOn] 报点+光传感器+接近光+九轴惯性传感器+报点包含压力信息
% [2018-PalmTouch] 电容屏数据
% [2019-BeyondUnint] 背面输入的轨迹
% tablet:
% [2006-PadUnint] 报点(当时不是电容屏)
% [2014-PenUnint] 电容屏,研究行为的时候用了许多的传感器,但是技术只依赖电容屏
% [2014-PalmRejection] 报点(利用到时序上的统计数据作为feature)
% [2013-TapBoard] 报点
% [2016-TapBoard2] 报点
% Tabletop:
% [2020-TabletopTouch] 报点+头动眼动信号

\subsubsection{Summary}

As table xx shows, we proposed a classification of unintentional touch rejection methods into two broad categories: \emph{universal method} and \emph{specific method}. We define \emph{universal methods} as those techniques that identifies and filters out unintentional touches over different scenarios. For \emph{universal methods}, unintentional touches refer to those touches that do not contribute to any interaction goal. We define \emph{specific methods} as those techniques that identifies unintentional touches in specific scenarios. For example, unintentional touches in the text entry task are those touches that do not represent typing intent. As table xx shows, the recognition rate of \emph{specific methods} are generally higher that the performance of \emph{universal methods}. We argue that this gap is significant. As the performances of \emph{universal methods} were not higher then 92\%, the lowest achievable error rate (Bayes error rate \cite{tumer1996estimating}) might be high. \emph{Universal methods} do not significantly change user's behavior, i.e., users were forced to perform unnatural and uncomfortable actions to avoid unintentional touches \cite{2014-PenMightier} [xx]. For comparison, the performances of \emph{specific methods} are much higher. Users are willing to change their behavior to acquire benefits from the techniques \cite{2013-TapBoard,2016-TapBoard2}.

Beside use scenario, the sensing ability of the devices also affect the recognition rate of unintentional touch. Several studies have shown that additional input channel such as xx, xx and xx significantly improve the accuracy. In this work, we used a device with strong sensing ability (force-sensitive touchscreen) to identify unintentional touch in a specific task (text entry). There was no doubt that we were able to reach a high recognition rate. In this situation, we were interested in a deeper question, which is the interrelationship between technique and user behavior. On the one hand, a perfect technique of rejecting unintentional touch will enable natural and relaxing user behavior on the touchscreen. On the other hand, a perfect technique should adapt to the natural user behavior. To our knowledge, previous studies did not analyze natural user behavior on a unintentional touch rejecting touchscreen. They either collected user data in a system with no feedback [xx,xx,xx] or emulational feedback [xx]. In this paper, we introduce a iterative progress to reveal the interrelationship between unintentional touch rejection technique and user behavior.

TapBoard \cite{2013-TapBoard} was similar to our work. We both investigate unintentional touch rejection on touchscreen keyboard. However, TapBoard used thresholds to distinguish between typing and unintentional touch, thus users need to adapt their behaviors to the technique. This could not provide natural and relaxing user experience [xx, xx]. Moreover, TapBoard suffered from modest accuracy (97\%) and high recognition delay (recognized at the touch up moment). In this paper, we cover TapBoard's shortage by considering the interrelationship between technique and user behavior.

\begin{table}[]
	\begin{tabular}{l|l|l}
		& universal method                                                                                             & specific method                                                                                             \\ \hline
		off-the-shelf sensors & \begin{tabular}[c]{@{}l@{}}xxx (xx.x\%) {[}xx{]}\\ xxx (xx.x\%) {[}xx{]}\\ xxx (xx.x\%) {[}xx{]}\end{tabular} & \begin{tabular}[c]{@{}l@{}}xxx (xx.x\%) {[}xx{]}\\ xxx (xx.x\%) {[}xx{]}\\ xxx (xx.x\%) {[}xx{]}\end{tabular} \\ \hline
		additional sensors    & \begin{tabular}[c]{@{}l@{}}xxx (xx.x\%) {[}xx{]}\\ xxx (xx.x\%) {[}xx{]}\\ xxx (xx.x\%) {[}xx{]}\end{tabular} & \begin{tabular}[c]{@{}l@{}}xxx (xx.x\%) {[}xx{]}\\ xxx (xx.x\%) {[}xx{]}\\ xxx (xx.x\%) {[}xx{]}\end{tabular}
	\end{tabular}
	\label{tab:related-work}
	\caption{Classification of unintentional touch rejection techniques.}
\end{table}

%如表格xx所示,我们用一个2*2的空间分类了上述相关工作。一般来说,传感器的能力越强,任务场景越窄,识别准确率就越高,TypeBoard在表格中的位置决定了它有很高的识别上限。这使得我们有机会探索一个全新的问题:在防误触算法十分强大的情况下,用户的行为会发生改变吗?据我们所知,还没有工作很好地回答过这个问题。有的实验设置在采集数据时不给出反馈[xx],有的实验则是给出了未经防误触算法过滤的反馈[xx],这都不能正确反映用户在鲁棒技术上的行为模型。在这份工作中,我们首次揭示了防误触能力和用户行为之间的相互影响,并通过迭代的方法,不断地采集用户数据、训练机器学习模型,最终实现了鲁棒的防误触算法。

%有一份叫TapBoard的工作和我们的论文TypeBoard很类似\cite{2013-TapBoard},我们研究的都是触屏打字场景下区分打字点击和误触。我们的工作和TapBoard相比最大的不同是对“误触”的不同定义:在TapBoard中,误触指的是触摸时间超过xx毫秒或者移动距离超过xx毫米的点击,用户需要主动适应这一设定;而在TypeBoard中,误触指的是“不用来表达按键意图的点击”,是技术主动去适应用户的行为。TapBoard有许多局限性:首先也是最重要的是,由于TapBoard要求用户主动适应技术,这无法形成最自然、舒适的用户行为[xx],也带来了一定的学习成本;第二,TapBoard采用阈值方法识别误触,其准确率有限,在用户主动适应的情况下只有大约97\%的准确率,这意味着用户行为大约1分钟就要被误识别一次;第三,TapBoard在手指抬起的瞬间才能识别点击意图,与正规键盘中手指点击瞬间就完成识别相比有很大的延迟,这会影响输入效率和用户体验[xx]。

\subsection{Benefits from Unintentional Touch Rejection}

Since unintentional touch affects the efficiency and naturalness of the touch interaction, a straightforward benefit from unintentional touch rejection is to improve the user experience. Beside this, we summarize other advantages of unintentional touch rejection as follows.

\subsubsection{Bridge the Gap Between Soft Keyboard and Physical Keyboard}

Though touchscreen devices are getting popular in the past ten years [xx], users still prefer to use personal computers and physical keyboards in the office work [xx]. This is because physical keyboards are better than touchscreen keyboards in usability [xx] and efficiency (xx-WPM vs. xx-WPM [xx]). Users can rest their fingers on physical keyboards, which is an important factor in the usability of a keyboard \cite{2013-TapBoard}. First, users avoid fatigue by resting their fingers in a long text entry task. Second, users align their fingers by touching the tactile landmarks on physical keyboards. This alignment behavior enable efficient touch type.

Our work support unintentional touch rejection on touchscreens, that is, users are able to rest their fingers on the touchscreen. This improvements has three advantages. The first is avoiding fatigue. Second, our studies show that the resting behavior improved the accuracy of touch position in each keystroke, because the users' fingers got closer to the desired keys. Third, we added stickers on the touchscreen keyboard to provide tactile landmarks and found it significantly improve the accuracy and efficiency. In real use, deformable touchscreen surfaces \cite{2011-Stimtac,2010-TeslaTouch,2011-Enhancing} and tangible keyboard objects on the touchscreen \cite{2008-Slap,2020-TouchFire} can provide such tactile landmarks. Thus, unintentional touch rejection does bridge the gap between soft keyboard and physical keyboard.

%虽然触屏设备在近十年的时间内迅速普及,但是人们在办公场景下还是会使用笔记本电脑和物理键盘进行文本输入,这是因为物理键盘的可用性和打字效率(xx-WPM vs. xx-WPM)都远远好于触屏设备上的软键盘。在物理键盘上,用户可以将手指休息在键盘上,这是物理键盘可用性的一个重要的特征:第一,用户将手指休息在键盘上时,可以避免长时间打字带来的疲劳[xx];第二,用户可以通过物理键盘上的触觉纹理来对齐他们的手指,从而实现高效的盲打。

%我们这份工作支持了触屏键盘上的防误触算法,也就是说,我们让用户可以将手指手指休息在键盘上,避免长时间输入带来的疲劳。我们在实验中发现,若用户习惯将手指休息在触碰键盘上,他每次打字点击的准确率会提升,这是因为在打字的时间间隙里,用户的手指离每个按键的距离更近了。更进一步的,我们在触屏键盘上通过贴纸模拟了触觉纹理,并发现触觉纹理能够显著提升用户打字的准确率和速度。在真实使用中,触觉纹理可以通过可变形触屏[xx]或者添加物理层[xx]的方法来提供。这说明,防误触算法的确能够有效缩小软键盘和物理键盘之间的差距。

%在防误触算法的支持下,我们可以通过可变形的触摸屏\cite{2011-Stimtac,2010-TeslaTouch,2011-Enhancing}或者在触屏上加装硬件膜\cite{2008-Slap,2020-TouchFire}的方法来提供触觉landmarks,从而支持触屏键盘上的盲打。

%如我们所知,物理键盘上的打字速度远远超过触屏键盘上的打字速度(xxWPM vs. xxWPM)。物理键盘上有release/touch/press三种状态,而触屏键盘上只有release/press两种状态。Kim等人\cite{2013-TapBoard}指出,物理键盘上多出来的touch状态是an important factor in the usability of a keyboard,因此在触屏键盘上增添touch状态能够有效地弥补触屏键盘和物理键盘之间的GAP。在触屏上区分打字点击和误触是增加touch状态的一个直接的想法:我们可以将触屏上的有意点击映射为物理键盘上的press状态,将触屏上的误触映射为物理键盘上的touch状态\cite{2013-TapBoard,2016-TapBoard2}。拥有touch状态的触屏键盘有多重好处:首先,用户可以将手指休息在触屏键盘上,降低手指悬空打字的疲劳\cite{2013-TouchDisplay};第二,因为用户可以轻轻触摸触屏而不触发打字,我们可以通过xx\cite{2011-Stimtac,2010-TeslaTouch,2011-Enhancing}或者在触屏上加装硬件膜\cite{2008-Slap,2020-TouchFire}的方法来提供触觉landmarks,从而支持触屏键盘上的盲打。

% 有不少工作通过提供touch状态来增强触屏键盘的能力,人们从硬件和软件两个层面来作出努力。在硬件层面上,TouchFire[TapBoard-30]和SLAP Widget[TapBoard-31]在触屏键盘上提供了一个tangible keyboard object,只有keystrokes会通过机械结果传输到触屏上,而其它的点击会被屏蔽。Tactus technology [TapBoard-29]开发了一款可以变形的触摸屏,可以动态地生成物理凸起。TeslaTouch [TapBoard-3] LATPaD [TapBoard-16], and STIMTAC [TapBoard-1]都展示了动态可变化的表面纹理,以上技术都可以为触屏打字用户提供tactile landmarks,从而支持触屏上的盲打。 在软件层面上,TapBoard采用简单的触摸时间和距离阈值来区分打字和误触,从而允许用户将手指休息在触摸屏上[xx],有利于降低用户在长时间触屏打字时的疲劳感\cite{2013-TouchDisplay}。

% [2013-TouchDisplay] Palm rejection (PR) technology may reduce shoulder loads by allowing the palms to rest on the display and increase productivity by registering the touched content and fingertips through the palms rather than shoulders.

% [2014-PenMightier] The digital devices prevented participants from interacting naturally because participants altered their behaviour to avoid making unintended, accidental markings. With the passive system, participants were “forced to write in an uncomfortable position to avoid the ‘palm touch’ screen” and “could not rest [their] palm on the display without disrupting it – highly unusable”. With the active system, participants were “more willing to interact because [they] could rest [their] palm on the surface with no problems” and “the Slate didn’t have the palm ‘touchy’ problems that the iPad did.

\subsubsection{Leveraging Unintentional Touch as Input Channels}

The goal of our work is filtering out unintentional touches on the touchscreen. However, there are some literature that tried to "use" unintentional touches as situational awareness contexts [xx] or explicit interactions [xx]. For situational awareness, Zhang et al. leveraged unintentional touches such as the palm touch to augment pen and touch interactions, e.g., providing "Palm Tools" that teleport to a hand-centric location when the user plants their palm on the screen while drawing \cite{2019-PenTouch}. For explicit interactions, Matulic el at. enriched tabletop interaction by considering whole hands as input \cite{2017-HandContact}. The algorithm detects seven different contact shapes with 91\% accuracy and can be used to trigger, parameterise, and dynamically control menu and tool widgets. PalmTouch is an additional input modality that differentiates between touches of fingers and the palm \cite{2018-PalmTouch}. PalmTouch can be used as a shortcut for improving reachability. TapBoard2 \cite{2016-TapBoard2} distinguishes typing from pointing using a threshold method, thereby unifying the keyboard and mouse control spaces. This depresses the burden of frequently switching between devices \cite{2006-TouchType}.

%我们的目标是去除触屏上的误触,从而支持流畅的文本输入法。然而,有的文献用另一个视角来看待误触,他们将误触利用为有用的信息,从而提供额外的输入通道。Matulic等人\cite{2017-HandContact}将手指交互拓展为包含手部形状的全手型交互。他们采用了一个基于计算机视觉的算法来在tabletop上识别hand-contact-shapes,能够以大约91\%的准确率识别检测七种不同的contact-shapes(such as fist, flat palm and spread hand)。 Zhang等人\cite{2019-PenTouch}提出利用多样误触信息来增强笔和触屏交互,例如根据大鱼际误触来确定持握手的位置,从而支持鲁棒的屏幕自动旋转功能。PalmTouch is \cite{2018-PalmTouch} an additional input modality that differentiates between touches of fingers and the palm. The use cases for PalmTouch include the use as a shortcut and for improving reachability. The authors have developed a model that differentiates between finger and palm touch with an accuracy of 99.53\% in realistic scenarios. 在触屏键盘上,pointing动作也算是误触的一种,TapBoard2\cite{2016-TapBoard2}通过阈值的方法区分了Typing和Pointing操作,从而将文本输入和光标操控统一到一个触摸屏当中,解决了频繁切换键鼠带来的低效问题\cite{2006-TouchType}。
