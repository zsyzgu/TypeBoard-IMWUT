\section{Related Work}

1. 触屏上的防误触算法

防误触的算法可以按两种方法分类,一是按正例的类型分类,二是按输入模态分类。

正例可以是广义的触摸输入,也可以是特定的任务,如打字、触屏笔写字等等;输入的模态可以是传统的触摸屏,也可以在触碰上加装压力、xx等信息。甚至是结合了肢体语言、眼动、头动等触屏之外的信息。

在这份工作中,我们考虑用带有压力的触摸屏,来实现打字任务下的防误触问题。一方面,这一问题是重要的,因为大多数触屏输入法还没有防误触算法,使得用户输入的效率和自然性受到影响;另一方面,这一收窄过的问题是理论上可解的,在我们的试验中,有xx.x\%的数据是可以人工标注的,只要我们使用机器学习去接近这一准确率,该技术就能实现。

与先前工作不同的是,我们这份工作首次考虑到了技术和用户行为之间的影响,有的工作在采集数据时不给出反馈,或者是给出了未经防误触算法过滤的反馈,这都不能正确反映用户在最终技术上的心理模型。在这份工作中,我们采用迭代式的方法,重复地采集用户数据和优化机器学习算法,最终总结出用户在一个“近乎完美防误触键盘”上打字的行为,并根据此用户行为,设计了一款“近乎完美防误触的键盘”。

TapBoard和我们的最像。但它有不少缺陷。另外,TapBoard在实验设计上有一个错误,他们通过一个基于誊写任务的对比实验,证明了TapBoard的输入效率和传统触屏键盘无异,这是以偏概全的。我们通过Pilot-study发现,在誊写等快速打字的任务中,用户很少将手指休息在触屏上,因此TapBoard无法证明用户在写日记等任务中不会因为手指的休息造成误触问题。

2. 触屏键盘上支持防误触的好处

(1) 有机会弥补触屏键盘上没有触摸状态的问题

物理键盘上有release/touch/press三态,而触屏键盘上只有release/press两态。touch状态在物理键盘上承担着重要的作用,具体有三点:1、物理键盘上的触觉纹理可以让用户不看键盘的情况下对齐手指,从而支持盲打;2、键帽能够给用户提供触觉反馈,确认一次点击操作,这不仅增强了用户体验,也显著地提升了打字速度;3、在物理键盘上,用户可以将手指休息在键位上,降低了疲劳程度。触屏键盘上touch状态的缺少,使得触屏上的打字效率显著低于物理键盘打字的效率。

有不少工作尝试弥补物理键盘和触屏键盘之间的gap。xx通过加装xx硬件,提供了纹理信息,达到了xx效果。xx通过加装xx硬件,提供了用户在打字时的触觉反馈,达到了xx的效果。xx采用简单的时间和距离阈值方法来区分打字和误触,从而允许用户将手指休息在触屏键盘上。然而,以上对触屏键盘的体验提升,都是以能够正确区分打字和误触为前提的。而还没有哪份工作做到了很高的识别准确率,这也是我们这份工作的重要动机。

(2) 在touchscreen上无缝区分keyboard和其它有意输入

如果能够在touchsreen上identify-typing-action,就可以仅使用触摸屏的同一块区域,同时支持文本输入和其它交互,如触摸屏控制鼠标pointing,又如其它的一些手势命令等等。TapBoard2能够有效地将打字的触摸事件和pointing的触摸事件正确区分,使得同一块触摸屏及支持了文本输入,又支持了触摸板控制鼠标。xx将手掌的误触作为输入通道。

还可以区分typing和stylus输入

[2014-PenUnint]

[2014-PenMightier]

[2014-PalmRejection]

[Exploring and Understanding Unintended Touch during Direct Pen Interaction]

(3) 将原本应该是误触的触碰利用为输入信息

Matulic et al. [2017-HandContact] extended hand interactions from fingertips to the whole hand in hand-shape based interaction. Tabletop interaction can be enriched by considering whole hands as input instead of only fingertips. We describe a
generalised, reproducible computer vision algorithm to recognise hand contact shapes, with support for arm rejection, as well as dynamic properties like finger movement
and hover. A controlled experiment shows the algorithm can detect seven different contact shapes (such as fist, flat palm and spread hand) with roughly 91\% average accuracy如果能将type点击事件和手掌的触摸事件区分开来,就可以支持hand contact shape所支持的交互方式。

Zhang et al. [77] proposed to leverage various hand postures such as using the palm to augment pen and touch interactions.

[2018-PalmTouch]

将大鱼际的误触作为模式切换。

We present PalmTouch, an additional input modality that differentiates between touches of fingers and the palm. We present different use cases for PalmTouch, including the
use as a shortcut and for improving reachability.  We have developed a model that differentiates between finger and palm touch with an accuracy of 99.53\% in realistic scenarios.

[??] 比如处理漂移问题。


