\section{Related Work}

\subsection{Unintentional Touch Rejection on Touch Screen}

在没有防误触的屏幕上操作时,用户需要小心谨慎地进行输入,这使得用户难以进行自然和放松的触摸输入\cite{2014-PenMightier, 2020-TabletopTouch}。为了提高用户触摸输入的体验,有大量的工作研究了防误触问题。在本子章节中,我们按照两种分类方式来列举和比较现有的工作,分别是按设备和传感器能力分类,和按“误触”的定义分类。

(1)按设备和传感器能力分类

我们可以根据设备和传感器能力来分类相关工作。由于防误触问题的重要性,在智能手机\cite{2012-IdentifyUnint,2015-GestureOn,2018-PalmTouch,2019-BeyondUnint}、平板电脑\cite{2006-PadUnint,2014-PenUnint,2014-PalmRejection,2013-TapBoard,2016-TapBoard2}和大桌面\cite{2020-TabletopTouch}等等触摸设备上,都有大量防误触的相关工作,其中大多数工作都是仅利用传统触摸设备中自带的报点数据\cite{2012-IdentifyUnint,2006-PadUnint,2014-PalmRejection,2013-TapBoard,2016-TapBoard2}、电容屏图像\cite{2018-PalmTouch,2014-PenUnint}来判断误触的。Matero等人研究了利用智能手机触屏报点信息来识别误触的可行性\cite{2012-IdentifyUnint}。他们在智能手机上采集了17名用户在swipe interactions in the home view, traditional phone call interaction, and general device handling这三种典型的应用场景下的触屏报点数据,并提出了六个与点击时长、落点位置和轨迹规律相关的数值作为filtering criteria,最终能rejected 76.6\%的误触点,同时只rejected了0.8\%的有意点击。Schwarz等人xx\cite{2014-PalmRejection},提出了一个概率性的触摸过滤器用于区分legitimate stylus and palm touches。They used spatiotemporal features and decision forest model to distinguish palm touches from stylus input。他们的系统可以将accidental palm inputs降低至0.016每次pen stroke,同时正确地识别了97.9\%的笔迹输入。PalmTouch\cite{2018-PalmTouch}是一种附加的输入模态,which可以区分手指输入和手掌输入,并利用手掌输入作为快捷键来提高手机系统的可达性。PalmTouch将整个电容屏的图像信息作为输入,并用卷积神经网络训练二分类模型,最终达到了99.5\%的准确率。这类工作的优势在于仅利用了现呈的设备和传感器,其算法能够马上应用在目前大部分手机和平板电脑上;其缺点是低准确率或是受限的应用场景。

另一部分工作在现有设备之上加入了其它的输入信道\cite{2015-GestureOn,2020-TabletopTouch,2001-PalmPressure,2019-BeyondUnint}。GestureOn\cite{2015-GestureOn}试图将黑屏时的手势输入和日常生活中的误触区分处理,这份工作广泛地利用了手机上所有能获取到的传感器数据,比如光敏传感器、接近光传感器、惯性传感器和带有压力的触屏报点信息。其中触屏上压力信息是技术上成熟,但还未完全普及的技术。融合了这些信息以后,GestureOn acquired 98.2\% precision and 97.6\% recall on detecting gestures from accidental touches。徐等人尝试解决大桌面上误触问题\cite{2020-TabletopTouch},它们将误触定义为一切不表达输入意图的触屏报点,在这份工作中,他们的防误触算法利用到了头动和眼动的信息,which在目前来说只能在实验室环境下获取,最后,这份工作达到了an F1 score of 91.3\%。以上的工作都在最常用的触屏设备上加上额外的传感器,能够不出意外地提升误触的识别准确率,tradeoff是不能马上应用在未经改装的手机上。

% Smartphone:`
% [2012-IdentifyUnint] 报点
% [2015-GestureOn] 报点+光传感器+接近光+九轴惯性传感器+报点包含压力信息
% [2018-PalmTouch] 电容屏数据
% [2019-BeyondUnint] 背面输入的轨迹
% tablet:
% [2006-PadUnint] 报点(当时不是电容屏)
% [2014-PenUnint] 电容屏,研究行为的时候用了许多的传感器,但是技术只依赖电容屏
% [2014-PalmRejection] 报点(利用到时序上的统计数据作为feature)
% [2013-TapBoard] 报点
% [2016-TapBoard2] 报点
% Tabletop:
% [2020-TabletopTouch] 报点+头动眼动信号

(2)按误触的定义分类

我们可以通过对误触的定义来分类相关工作种触屏上的防误触算法。有少量的文献\cite{2012-IdentifyUnint,2020-TabletopTouch}和大量的专利\cite{2016-Classification,2006-PadUnint,2013-System,2013-Precluding,2015-TouchScreen}将unintentional touch定义为不代表任何输入意图的触摸点\cite{2020-TabletopTouch},这一类工作研究的是设备上通用的防误触问题,不区分具体的任务。也有工作将误触称为unwanted touch或accidental touch\cite{2015-GestureOn,2012-IdentifyUnint}。这一类工作的优点在于适用于广泛的应用场景和任务,其tradeoff是相对降低的识别准确率,例如手机上利用触屏报点识别误触的recall只有76.6\%\cite{2012-IdentifyUnint},又如在大桌面上结合了触屏报点、头动、眼动等信息判断误触,其准去率(F1 score)也只有91.3\%。这一类工作有两大致命的缺点,一是准确率很低,这一低准确率可能暗示着正常点击和“不代表任何输入意图的点击”在现有的传感器数据下有时候是不可区分的,而不会future-work中所说的随着机器学习的进步而得到一个可用的水平;二是他们无法通过简短的几个lab-study证明算法的普适性,而只能在作为limitation进行讨论。

在学术界,更多的工作研究了限定的场景和任务下的防误触问题。有的工作专门讨论了触屏键盘的防误触问题\cite{2013-TapBoard,2016-TapBoard2},在这一类工作中,误触的定义是除了“有意的打字点击”以外的触摸点。TapBoard\cite{2013-TapBoard}将短暂的tap动作视为有意的打字点击,而将其它触摸事件视为误触点,并通过阈值的方法区分它们,结果是用户可以适应这种技术,在TapBoard上打字的效率与在普通触屏上打字无异,而用户平时可以将手指轻轻放在屏幕上休息。TapBoard2\cite{2016-TapBoard2}我还没有看明白呢!有的工作着力于区分触屏上的电容笔点击和手掌的误触,基于报点和电容屏的数据,这些工作采用时空特征和简单的机器学习,达到了xx.x\%以上的准确率,具体也是还没看呢!\cite{2013-PalmInput,2014-PenUnint,2014-PalmRejection}。还有的工作在防误触的情况下识别了特定的触摸手势。例如PalmTouch\cite{2018-PalmTouch}以99.5\%的准确率区分了手指输入和手掌输入,从而将手掌输入作为特定功能的快捷方式。GestureOn\cite{2015-GestureOn}以97.9的准确率(F1 score)区分了黑屏时的手势输入和日常生活中的误触,也是为了表达快捷方式。以上这一类特定场景下的误触识别问题,其准确率往往比较高,更接近实用的水平,作为tradeoff是仅限于特定的应用场景和任务。

(3)我们这份工作在相关工作中的位置

如表格xx所示,我们用一个2*2的空间分类了上述相关工作。一般来说,传感器的能力越强,任务场景越窄,识别准确率就越高,TypeBoard在表格中的位置决定了它有很高的识别上限。这使得我们有机会探索一个全新的问题:在防误触算法十分强大的情况下,用户的行为会发生改变吗?据我们所知,还没有工作很好地回答过这个问题。有的实验设置在采集数据时不给出反馈[xx],有的实验则是给出了未经防误触算法过滤的反馈[xx],这都不能正确反映用户在鲁棒技术上的行为模型。在这份工作中,我们首次揭示了防误触能力和用户行为之间的相互影响,并通过迭代的方法,不断地采集用户数据、训练机器学习模型,最终实现了鲁棒的防误触算法。

【表格:触屏防误触算法的2*2分类,一维是传感器是否现呈,另一维是任务场景是否限定】

有一份叫TapBoard的工作和我们的论文TypeBoard很类似\cite{2013-TapBoard},我们研究的都是触屏打字场景下区分打字点击和误触。我们的工作和TapBoard相比最大的不同是对“误触”的不同定义:在TapBoard中,误触指的是触摸时间超过xx毫秒或者移动距离超过xx毫米的点击,用户需要主动适应这一设定;而在TypeBoard中,误触指的是“不用来表达按键意图的点击”,是技术主动去适应用户的行为。TapBoard有许多局限性:首先也是最重要的是,由于TapBoard要求用户主动适应技术,这无法形成最自然、舒适的用户行为[xx],也带来了一定的学习成本;第二,TapBoard采用阈值方法识别误触,其准确率有限,在用户主动适应的情况下只有大约97\%的准确率,这意味着用户行为大约1分钟就要被误识别一次;第三,TapBoard在手指抬起的瞬间才能识别点击意图,与正规键盘中手指点击瞬间就完成识别相比有很大的延迟,这会影响输入效率和用户体验[xx];第四,TapBoard声称它有两大好处,一是用户可以休息手指,二是其输入效率与原本键盘一样。这个结论是错误的,TapBoard的实验只证明了用户在边写边说的实验中会休息手指,而在誊写任务下输入效率与原本触屏键盘无异,然而根据我们的实验结果,用户在誊写任务下并不会休息手指。因此以上两个好处并不同时成立,也不具有跨任务的普适性。相比之下,我们的工作TypeBoard克服了TapBoard所有的局限性。

% 如表格xx所示,我们用一个2*2的空间来展示了我们这份工作的位置,其中一维是传感器能力,另一维是任务场景。可以看到,我们的工作是少有的,两个维度上都倾向于“提高精度”的,这使得我们有机会获得一个很高的准确率,从而探索一个全新的问题:在防误触算法十分鲁棒的前提下,用户的行为会发生变化吗?据我们所知,还没有工作很好地回答过这个问题,有的实验设置在采集数据时不给出反馈[xx],有的实验是给出了未经防误触算法过滤的反馈[xx],这都不能正确反映用户在最终技术上的心理模型。在这份工作中,我们首次揭示了防误触能力和用户行为之间的相互影响,并通过迭代的方法,不断地采集用户数据、训练机器学习模型,最终同时一个十分鲁棒的防误触算法,并发现用户在其上的打字行为发生本质的改变。

\subsection{Benefit from Unintentional Touch Rejection}

(1) 弥补触屏键盘和物理键盘之间的GAP

众所周知,物理键盘上的打字速度远远超过触屏键盘上的打字速度(xx-WPM vs xx-WPM)。xx等人[xx]指出,物理键盘上有release/touch/press三种状态,而触屏键盘上只有release/press两种状态。物理键盘上多出来的touch(but not press)状态是其支持高校文本输入的重要特征,因为它为物理键盘带来了三大好处:1、物理键盘上提供的tactile landmark可以让用户在eye-free的情况下align-finger,从而使能了盲打,大大提高了打字速度;2、键帽给用户提供了触觉反馈,这一反馈可以让用户确认点击的生效,真正地提高输入效率[xx],也能提高用户对点击时间的掌控[xx];3、在物理键盘上,用户可以将手指休息在键位上,降低了疲劳程度[xx]。

有不少工作通过提供touch状态来增强触屏键盘的能力,人们从硬件和软件两个层面来作出努力。在硬件层面上,TouchFire[TapBoard-30]和SLAP Widget[TapBoard-31]在触屏键盘上提供了一个tangible keyboard object,只有keystrokes会通过机械结果传输到触屏上,而其它的点击会被屏蔽。Tactus technology [TapBoard-29]开发了一款可以变形的触摸屏,可以动态地生成物理凸起。TeslaTouch [TapBoard-3] LATPaD [TapBoard-16], and STIMTAC [TapBoard-1]都展示了动态可变化的表面纹理,以上技术都可以为触屏打字用户提供tactile landmarks,从而支持触屏上的盲打。 在软件层面上,TapBoard采用简单的触摸时间和距离阈值来区分打字和误触,从而允许用户将手指休息在触摸屏上[xx],有利于降低用户在长时间触屏打字时的疲劳感\cite{2013-TouchDisplay}。

在触碰上区分打字点击和误触是增加touch状态的一个非常直接的想法,我们可以将有意的打字点击映射为物理键盘上的press状态,将误触映射为物理键盘上的touch状态[xx, xx]。

[2013-TouchDisplay]
Palm rejection (PR) technology may reduce shoulder loads by allowing the palms to rest on the display and increase productivity by registering the touched content and fingertips through the palms rather than shoulders.

[2014-PenMightier]
The digital devices prevented participants from interacting naturally because participants altered their behaviour to avoid making unintended, accidental markings. With the passive system, participants were “forced to write in an uncomfortable position to avoid the ‘palm touch’ screen” and “could not rest [their] palm on the display without disrupting it –
highly unusable”. With the active system, participants were “more willing to interact because [they] could rest [their] palm on the surface with no problems” and “the Slate didn’t have the palm ‘touchy’ problems that the iPad did.

(2) 在touchscreen上无缝区分keyboard和其它有意输入

如果能够在touchsreen上identify-typing-action,就可以仅使用触摸屏的同一块区域,同时支持文本输入和其它交互,如触摸屏控制鼠标pointing[xx],又如其它的一些手势命令[xx]等等。TapBoard2能够有效地将打字的触摸事件和pointing的触摸事件正确区分,使得同一块触摸屏及支持了文本输入,又支持了触摸板控制鼠标。xx将手掌的误触作为输入通道。

还可以区分typing和stylus输入

[2014-PenUnint]
[2014-PenMightier]
[2014-PalmRejection]

[Exploring and Understanding Unintended Touch during Direct Pen Interaction]

(3) 将原本应该是误触的触碰利用为输入信息

有的相关工作将误触作为有用的输入信息,从而提供额外的输入通道。

Matulic et al. [2017-HandContact] extended hand interactions from fingertips to the whole hand in hand-shape based interaction. Tabletop interaction can be enriched by considering whole hands as input instead of only fingertips. We describe a
generalised, reproducible computer vision algorithm to recognise hand contact shapes, with support for arm rejection, as well as dynamic properties like finger movement
and hover. A controlled experiment shows the algorithm can detect seven different contact shapes (such as fist, flat palm and spread hand) with roughly 91\% average accuracy如果能将type点击事件和手掌的触摸事件区分开来,就可以支持hand contact shape所支持的交互方式。

Zhang et al. [77] proposed to leverage various hand postures such as using the palm to augment pen and touch interactions.

[2018-PalmTouch]

将大鱼际的误触作为模式切换。

We present PalmTouch, an additional input modality that differentiates between touches of fingers and the palm. We present different use cases for PalmTouch, including the
use as a shortcut and for improving reachability.  We have developed a model that differentiates between finger and palm touch with an accuracy of 99.53\% in realistic scenarios.

[??] 比如处理漂移问题。

